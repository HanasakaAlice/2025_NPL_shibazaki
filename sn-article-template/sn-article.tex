%Version 3.1 December 2024
% See section 11 of the User Manual for version history
%
%%%%%%%%%%%%%%%%%%%%%%%%%%%%%%%%%%%%%%%%%%%%%%%%%%%%%%%%%%%%%%%%%%%%%%
%%                                                                 %%
%% Please do not use \input{...} to include other tex files.       %%
%% Submit your LaTeX manuscript as one .tex document.              %%
%%                                                                 %%
%% All additional figures and files should be attached             %%
%% separately and not embedded in the \TeX\ document itself.       %%
%%                                                                 %%
%%%%%%%%%%%%%%%%%%%%%%%%%%%%%%%%%%%%%%%%%%%%%%%%%%%%%%%%%%%%%%%%%%%%%

%%\documentclass[referee,sn-basic]{sn-jnl}% referee option is meant for double line spacing

%%=======================================================%%
%% to print line numbers in the margin use lineno option %%
%%=======================================================%%

%%\documentclass[lineno,pdflatex,sn-basic]{sn-jnl}% Basic Springer Nature Reference Style/Chemistry Reference Style

%%=========================================================================================%%
%% the documentclass is set to pdflatex as default. You can delete it if not appropriate.  %%
%%=========================================================================================%%

%%\documentclass[sn-basic]{sn-jnl}% Basic Springer Nature Reference Style/Chemistry Reference Style

%%Note: the following reference styles support Namedate and Numbered referencing. By default the style follows the most common style. To switch between the options you can add or remove “Numbered” in the optional parenthesis. 
%%The option is available for: sn-basic.bst, sn-chicago.bst%  
 
%%\documentclass[pdflatex,sn-nature]{sn-jnl}% Style for submissions to Nature Portfolio journals
%%\documentclass[pdflatex,sn-basic]{sn-jnl}% Basic Springer Nature Reference Style/Chemistry Reference Style
\documentclass[pdflatex,sn-mathphys-num]{sn-jnl}% Math and Physical Sciences Numbered Reference Style
\usepackage{array}
%%\documentclass[pdflatex,sn-mathphys-ay]{sn-jnl}% Math and Physical Sciences Author Year Reference Style
%%\documentclass[pdflatex,sn-aps]{sn-jnl}% American Physical Society (APS) Reference Style
%%\documentclass[pdflatex,sn-vancouver-num]{sn-jnl}% Vancouver Numbered Reference Style
%%\documentclass[pdflatex,sn-vancouver-ay]{sn-jnl}% Vancouver Author Year Reference Style
%%\documentclass[pdflatex,sn-apa]{sn-jnl}% APA Reference Style
%%\documentclass[pdflatex,sn-chicago]{sn-jnl}% Chicago-based Humanities Reference Style

%%%% Standard Packages
%%<additional latex packages if required can be included here>

\usepackage{graphicx}%
\usepackage{multirow}%
\usepackage{amsmath,amssymb,amsfonts}%
\usepackage{amsthm}%
\usepackage{mathrsfs}%
\usepackage[title]{appendix}%
\usepackage{xcolor}%
\usepackage{textcomp}%
\usepackage{manyfoot}%
\usepackage{booktabs}%
\usepackage{algorithm}%
\usepackage{algorithmicx}%
\usepackage{algpseudocode}%
\usepackage{listings}%

\usepackage{bm}
\usepackage{amsmath,amssymb}
\usepackage{booktabs} 

% in preamble
\usepackage{amsthm}
\usepackage[most]{tcolorbox}

% --- theorem-like (italic) ---
\theoremstyle{plain}
\newtheorem{thm}{Theorem}[section]
\newtheorem{lem}[thm]{Lemma}
\newtheorem{prop}[thm]{Proposition}
\newtheorem{cor}[thm]{Corollary}

% --- definitions (upright) ---
\theoremstyle{definition}
\newtheorem{problem}{Problem Setting}[section]
\newtheorem{dfn}[thm]{Definition}
\newtheorem{ass}[thm]{Assumption}

% --- remarks (upright, small heading) ---
\theoremstyle{remark}
\newtheorem{rem}[thm]{Remark}



%%%%

%%%%%=============================================================================%%%%
%%%%  Remarks: This template is provided to aid authors with the preparation
%%%%  of original research articles intended for submission to journals published 
%%%%  by Springer Nature. The guidance has been prepared in partnership with 
%%%%  production teams to conform to Springer Nature technical requirements. 
%%%%  Editorial and presentation requirements differ among journal portfolios and 
%%%%  research disciplines. You may find sections in this template are irrelevant 
%%%%  to your work and are empowered to omit any such section if allowed by the 
%%%%  journal you intend to submit to. The submission guidelines and policies 
%%%%  of the journal take precedence. A detailed User Manual is available in the 
%%%%  template package for technical guidance.
%%%%%=============================================================================%%%%

%% as per the requirement new theorem styles can be included as shown below
\theoremstyle{thmstyleone}%
\newtheorem{theorem}{Theorem}%  meant for continuous numbers
%%\newtheorem{theorem}{Theorem}[section]% meant for sectionwise numbers
%% optional argument [theorem] produces theorem numbering sequence instead of independent numbers for Proposition
\newtheorem{proposition}[theorem]{Proposition}% 
%%\newtheorem{proposition}{Proposition}% to get separate numbers for theorem and proposition etc.

\theoremstyle{thmstyletwo}%
\newtheorem{example}{Example}%
\newtheorem{remark}{Remark}%

\theoremstyle{thmstylethree}%
\newtheorem{definition}{Definition}

\raggedbottom
%%\unnumbered% uncomment this for unnumbered level heads

\begin{document}

\title[Article Title]{Accurate and Robust Positive-Unlabeled Learning against Adversarial Perturbations}

%%=============================================================%%
%% GivenName	-> \fnm{Joergen W.}
%% Particle	-> \spfx{van der} -> surname prefix
%% FamilyName	-> \sur{Ploeg}
%% Suffix	-> \sfx{IV}
%% \author*[1,2]{\fnm{Joergen W.} \spfx{van der} \sur{Ploeg} 
%%  \sfx{IV}}\email{iauthor@gmail.com}
%%=============================================================%%

\author*[1]{\fnm{Ryo} \sur{Shibazaki}}\email{ryo.shibazaki0517@chiba-u.jp}

\author[2]{\fnm{Kazuhiko} \sur{Kawamoto}}\email{kawa@faculty.chiba-u.jp}

\author[2]{\fnm{Hiroshi} \sur{Kera}}\email{kera@chiba-u.jp}

\affil*[1]{\orgdiv{Graduate School of Science and Engineering}, \orgname{Chiba University},
\orgaddress{\street{1-33 Yayoich\={o}, Inage-ku}, \city{Chiba-shi}, \postcode{263-8522}, \state{Chiba}, \country{Japan}}}

\affil[2]{\orgdiv{Graduate School of Informatics}, \orgname{Chiba University},
\orgaddress{\street{1-33 Yayoich\={o}, Inage-ku}, \city{Chiba-shi}, \postcode{263-8522}, \state{Chiba}, \country{Japan}}}

%%==================================%%
%% Sample for unstructured abstract %%
%%==================================%%

\abstract{Labeling costs are high in domains such as medical image analysis, where Positive and Unlabeled (PU) learning, which trains using only positive and unlabeled data, is effective. Medical images often contain small perturbations due to sensor noise and variations in acquisition conditions, which can cause a classifier to misclassify images that should be positive as negative. Therefore, in settings where even minor misclassifications may lead to critical misdiagnoses, high robustness is required.
In this study, we focus on adversarial perturbations, which are known as worst-case noise among such perturbations, and aim to improve robustness within the PU learning framework. However, directly applying standard adversarial training methods to PU learning often severely degrades standard accuracy, making the trade-off between robustness and standard accuracy more pronounced. To address this issue, we propose PU-TRADES, a new learning method that extends the TRADES framework and integrates it with PU learning. Our method introduces label-independent adversarial perturbations and optimizes the balance between robustness and standard accuracy by combining a PU loss with a Kullback–Leibler loss.
Furthermore, we theoretically derive an upper bound on the estimation error for the proposed loss and clarify conditions under which PU learning can outperform supervised learning when the number of unlabeled samples is sufficiently large. Finally, experiments on multiple benchmark datasets and a medical imaging dataset demonstrate that the proposed method provides an effective framework for robust learning in PU settings.}

%%================================%%
%% Sample for structured abstract %%
%%================================%%

% \abstract{\textbf{Purpose:} The abstract serves both as a general introduction to the topic and as a brief, non-technical summary of the main results and their implications. The abstract must not include subheadings (unless expressly permitted in the journal's Instructions to Authors), equations or citations. As a guide the abstract should not exceed 200 words. Most journals do not set a hard limit however authors are advised to check the author instructions for the journal they are submitting to.

% \textbf{Methods:} The abstract serves both as a general introduction to the topic and as a brief, non-technical summary of the main results and their implications. The abstract must not include subheadings (unless expressly permitted in the journal's Instructions to Authors), equations or citations. As a guide the abstract should not exceed 200 words. Most journals do not set a hard limit however authors are advised to check the author instructions for the journal they are submitting to.

% \textbf{Results:} The abstract serves both as a general introduction to the topic and as a brief, non-technical summary of the main results and their implications. The abstract must not include subheadings (unless expressly permitted in the journal's Instructions to Authors), equations or citations. As a guide the abstract should not exceed 200 words. Most journals do not set a hard limit however authors are advised to check the author instructions for the journal they are submitting to.

% \textbf{Conclusion:} The abstract serves both as a general introduction to the topic and as a brief, non-technical summary of the main results and their implications. The abstract must not include subheadings (unless expressly permitted in the journal's Instructions to Authors), equations or citations. As a guide the abstract should not exceed 200 words. Most journals do not set a hard limit however authors are advised to check the author instructions for the journal they are submitting to.}

\keywords{positive-unlabeled learning, adversarial robustness, risk estimation, empirical risk minimization}


%%\pacs[JEL Classification]{D8, H51}

%%\pacs[MSC Classification]{35A01, 65L10, 65L12, 65L20, 65L70}

\maketitle

% ========================================================================
% Sections (each in separate file, like Appendix)
% ========================================================================
\section{Introduction} \label{ch:intro}

% TODO
% Refer to Fig.~\ref{fig:cil}.
% Add an explanation of class-incremental learning for the master's thesis version.

In recent years, machine learning has achieved remarkable success across a wide range of tasks, driven by advances in large-scale data and high-capacity models. However, in real-world applications, it is often difficult to obtain high-quality ...; moreover, in settings where the training set contains only positive and unlabeled data, one must learn from incomplete information in which the unlabeled set is a mixture of true positives and negatives. Therefore, unlike fully supervised learning, PU learning requires careful risk estimation and additional techniques to stabilize training.

Furthermore, in practical applications, robustness is an essential requirement in addition to label scarcity. In particular, medical images and sensor data are affected by variations in acquisition conditions, noise, and device ... We focus on adversarial perturbations\cite{Goodfellow2014ExplainingAH} and aim to improve robustness within the PU learning framework.

However, in preliminary experiments, directly applying adversarial training\cite{Madry2017TowardsDL}—a standard approach to enhancing resilience against adversarial perturbations—to PU learning can substantially degrade classification performance on clean data.
This is because, although unlabeled data contain a mixture of true positives and negatives, they are treated uniformly as negatives in the loss.
As a result, the objectives of adversarial perturbation optimization and classification become misaligned, which destabilizes the updates associated with the unlabeled term. Figure~\ref{fig:cil} illustrates the relationship between clean accuracy and adversarial accuracy when PGD-based adversarial training is naively applied to PU learning.
In particular, on CIFAR-10, adversarial accuracy improves while clean accuracy drops significantly. These observations indicate that, under the PU setting, introducing adversarial training may impair clean accuracy, which remains a key challenge.

In this study, we extend the TRADES framework\cite{zhang2019theoretically}, a representative adversarial training method, and propose a new learning method, PU-TRADES, by integrating it with PU learning.
Our approach handles adversarial perturbations in a label-independent manner and combines a PU loss with a Kullback--Leibler loss, aiming to improve robustness while suppressing degradation in clean accuracy.



\begin{figure}[t]
  \centering

  % --- scatter ---
  % 本番用: final_clean_scatter.pdf を sn-article-template フォルダに置く
  \includegraphics[width=0.92\linewidth]{final_clean_scatter.pdf}

  \vspace{1.0ex}

  % --- table ---
  \small
  \setlength{\tabcolsep}{5pt}
  \renewcommand{\arraystretch}{1.15}
  \begin{tabular}{l cc c cc c}
    \hline
     & \multicolumn{3}{c}{FashionMNIST} & \multicolumn{3}{c}{CIFAR-10} \\
     \cmidrule(lr){2-4}\cmidrule(lr){5-7}
    Method & Clean & Adv & $\Delta$Clean & Clean & Adv & $\Delta$Clean \\
    \hline
    uPU      & 0.937 & 0.190 & --     & 0.850 & 0.114 & -- \\
    nnPU     & \textbf{0.948} & 0.001 & --     & \textbf{0.887} & 0.001 & -- \\
    uPU+PGD  & 0.935 & 0.843 & -0.002 & 0.802 & 0.714 & -0.048 \\
    nnPU+PGD & 0.930 & 0.860 & -0.018 & 0.732 & 0.686 & \textbf{-0.155} \\
    \hline
  \end{tabular}

  \caption{Performance changes when PGD-based adversarial training is naively applied to PU learning (FashionMNIST / CIFAR-10).
  Left: a scatter plot showing the relationship between clean accuracy and adversarial accuracy for each method.
  Right: a numerical summary and the change in clean accuracy, $\Delta$Clean, before and after applying PGD (difference from the corresponding PU method without PGD).
  While PGD substantially improves adversarial accuracy, it degrades clean accuracy; this degradation is particularly pronounced for nnPU on CIFAR-10, where clean accuracy drops from $0.887$ to $0.732$ ($\Delta$Clean$=-0.155$).}
  \label{fig:cil}
\end{figure}



To validate the effectiveness of the proposed method, we conducted experiments on multiple benchmark datasets and medical imaging data. We evaluated both clean accuracy and ... and compared them with baseline methods. The results show that the proposed method substantially improves robustness to adversarial perturbations while maintaining accuracy on clean data.

In addition, we performed a theoretical analysis to better understand robustness in PU learning. Specifically, under binary classification with linear classifiers and adversarial perturbations, we derived upper bounds on the estimation error of the risks minimized by supervised learning and PU learning, thereby clarifying conditions under which PU learning can be advantageous over supervised learning. These conditions are consistent with practical scenarios, and they suggest that simply increasing the amount of unlabeled data can potentially achieve higher robustness than supervised learning.

Our contributions are summarized as follows.
\begin{itemize}
\item We propose a new learning framework (PU-TRADES) that integrates TRADES-style regularization into PU learning, improving robustness to adversarial perturbations while maintaining classification accuracy on clean samples.
\item Assuming linear classifiers under adversarial perturbations, we derive upper bounds on the estimation error of the risks minimized by supervised learning and PU learning, and theoretically identify conditions under which PU learning becomes more favorable than supervised learning.
\item Through experiments on benchmark datasets and medical imaging data, we empirically demonstrate that the proposed method acquires robustness to adversarial samples while preserving clean accuracy.
\end{itemize}

% ========================================================================


\section{Related Work} \label{ch:related}
% 2--3 pages
% Add a bit more detail on each method ...

In this chapter, we review prior studies related to this work, focusing on (i) Positive-Unlabeled (PU) learning and (ii) adversarial training. We then summarize existing research that combines PU learning with adversarial robustness.

\subsection{Positive-Unlabeled (PU) Learning}
PU learning is a classification framework in which only positive-labeled and unlabeled data are available. A representative line of work is risk-estimation-based PU learning, which constructs an (unbiased) estimator of the supervised classification risk using the class prior and the mixture structure of the unlabeled set.
In particular, uPU (unbiased PU learning) estimates the true risk without bias, while nnPU (non-negative PU learning) introduces a non-negativity constraint to prevent the empirical risk from becoming negative, thereby mitigating overfitting.
Many extensions have also been proposed, including methods that exploit high-confidence samples from the unlabeled set and approaches that incorporate various correction mechanisms.

\subsection{Adversarial Training}
Adversarial examples are inputs that are intentionally perturbed to cause a model to misclassify, and they have attracted extensive attention as a major threat to machine learning systems.
A standard defense is adversarial training, which improves robustness by training the model on adversarially perturbed samples. Representative methods such as PGD-based adversarial training can be formulated as a min--max optimization problem consisting of an outer minimization (parameter optimization) and an inner maximization (perturbation generation).
In addition, TRADES \cite{zhang2019theoretically} is a prominent method that introduces a regularization term based on the Kullback--Leibler divergence between the model outputs on clean and perturbed inputs, aiming to balance clean accuracy and adversarial robustness.

% \subsection{PU Learning with Adversarial Robustness}
% Several recent studies have explored robust learning under weak supervision, including settings related to PU learning. However, naively applying adversarial training to PU learning is non-trivial, because the unlabeled set contains a mixture of true positives and negatives, which can lead to unstable updates and substantial degradation in clean accuracy.
% In this study, we extend the TRADES framework to the PU learning setting and propose a method that achieves strong robustness under the constraints inherent to PU learning.


\section{Preliminaries}\label{ch:preliminaries}

  In this section, we introduce PU learning, adversarial examples and representative attacks, and adversarial training.
  Hereafter, we refer to learning a binary classifier from fully labeled positive and negative data as Positive--Negative (PN) learning.

\subsection{Positive-Unlabeled (PU) Learning}
We denote the input space by $\mathcal{X} \subseteq \mathbb{R}^d$ and the label space by $\mathcal{Y}=\{-1,+1\}$.
Let $p(\bm{x}, y)$ be the joint distribution over $(\mathcal{X},\mathcal{Y})$.
Let the total number of samples be $n \in \mathbb{N}$, and
let $n_{\mathrm{P}}$ and $n_{\mathrm{N}}$ denote the numbers of positive (P) and negative (N) samples, respectively.
Each set is represented as follows:
\begin{equation}
\begin{aligned}
\mathscr{X}_{\mathrm{P}} &= \{\boldsymbol{x}_i^{\mathrm{P}}\}_{i=1}^{n_{\mathrm{P}}}
\stackrel{\text{i.i.d.}}{\sim} p_{\mathrm{P}}(\boldsymbol{x}), \\
\mathscr{X}_{\mathrm{N}} &= \{\boldsymbol{x}_i^{\mathrm{N}}\}_{i=1}^{n_{\mathrm{N}}}
\stackrel{\text{i.i.d.}}{\sim} p_{\mathrm{N}}(\boldsymbol{x}).
\end{aligned}
\end{equation}
Here, $p_{\mathrm{P}}(\bm{x})$ and $p_{\mathrm{N}}(\bm{x})$ denote the class-conditional densities for the positive and negative classes, respectively.
The full dataset $\mathscr{X}=\mathscr{X}_{\mathrm{P}} \cup \mathscr{X}_{\mathrm{N}}$ is written as
\begin{equation}
\begin{aligned}
\mathscr{X} &= \{\boldsymbol{x}_i\}_{i=1}^{n} \stackrel{\text{i.i.d.}}{\sim} p(\boldsymbol{x}),\\
p(\boldsymbol{x}) &= \pi_{\mathrm{P}}\, p_{\mathrm{P}}(\boldsymbol{x}) + \pi_{\mathrm{N}}\, p_{\mathrm{N}}(\boldsymbol{x}),
\end{aligned}
\end{equation}
where $\pi_{\mathrm{P}}=p(y=+1)$ and $\pi_{\mathrm{N}}=...p(y=-1)$ are the class priors satisfying $\pi_{\mathrm{P}}+\pi_{\mathrm{N}}=1$.

In PU learning, the training set consists of positive (P) samples and unlabeled (U) samples.
Since the marginal distribution of unlabeled data is $p_{\mathrm{U}}(\bm{x})=\pi_{\mathrm{P}} p_{\mathrm{P}}(\bm{x})+\pi_{\mathrm{N}} p_{\mathrm{N}}(\bm{x})$, 
letting $n_{\mathrm{U}}$ be the number of unlabeled samples, the unlabeled set is given by
\begin{align}
\mathscr{X}_{\mathrm{U}}
&= \{\boldsymbol{x}_i^{\mathrm{U}}\}_{i=1}^{n_{\mathrm{U}}}
\stackrel{\text{i.i.d.}}{\sim} p_{\mathrm{U}}(\boldsymbol{x})
= p(\boldsymbol{x}).
\end{align}
That is, the unlabeled samples are drawn i.i.d. from the marginal distribution of inputs, which is a mixture of positive and negative class-conditional distributions.

\noindent\textbf{Unbiased PU Learning (uPU).}
uPU assumes that the positive class prior $\pi_{\mathrm{P}}$ is known and estimates the negative risk indirectly from the unlabeled data.
Specifically, it minimizes the following empirical risk:
\begin{align}
\widehat{R}_{\mathrm{uPU}}(g)
&=
\frac{\pi_{\mathrm{P}}}{n_{\mathrm{P}}} \sum_{i=1}^{n_{\mathrm{P}}}
\tilde{\ell}\!\left(g\!\left(\boldsymbol{x}_i^{\mathrm{P}}\right), +1\right)
\;+\;
\frac{1}{n_{\mathrm{U}}} \sum_{i=1}^{n_{\mathrm{U}}}
\ell\!\left(g\!\left(\boldsymbol{x}_i^{\mathrm{U}}\right), -1\right).
\end{align}
Here, the composite loss $\tilde{\ell}(g(\boldsymbol{x}), y)$ is defined by
$\tilde{\ell}(g(\boldsymbol{x}), y)
=\ell(g(\boldsymbol{x}), y)-\ell(g(\boldsymbol{x}), -y)$.


\noindent\textbf{Non-Negative PU Learning (nnPU).}
nnPU was introduced to address the overfitting issue in uPU, where the empirical risk can take negative values\cite{Kiryo2017PositiveUnlabeledLW}.
Specifically, when the estimated term involving the negative risk in PU learning becomes negative, nnPU clips its negative contribution to zero, yielding a non-negative risk estimator that keeps the overall estimate bounded below by $0$.
This modification prevents the empirical risk from diverging to negative values during empirical minimization and enables stable training.
The empirical risk of nnPU is defined as follows:
\begin{equation}
\begin{aligned}
\widehat{R}_{\mathrm{nnPU}}(g)
= {} & \frac{\pi_{\mathrm{P}}}{n_{\mathrm{P}}} \sum_{i=1}^{n_{\mathrm{P}}}
\ell\left(g\left(\boldsymbol{x}_i^{\mathrm{P}}\right),+1\right) \\
& + \max\Biggl\{0\,,
- \frac{\pi_{\mathrm{P}}}{n_{\mathrm{P}}} \sum_{i=1}^{n_{\mathrm{P}}}
\ell\left(g\left(\boldsymbol{x}_i^{\mathrm{P}}\right),-1\right)
+ \frac{1}{n_{\mathrm{U}}} \sum_{i=1}^{n_{\mathrm{U}}}
\ell\left(g\left(\boldsymbol{x}_i^{\mathrm{U}}\right),-1\right)
\Biggr\}.
\end{aligned}
\end{equation}


\subsection{Adversarial Examples} \label{AE}


  Adversarial examples are inputs constructed by adding small, carefully designed perturbations to an image that a classifier originally classified correctly, thereby intentionally causing misclassification...w2014ExplainingAH. For example, Fig.~\ref{fig:advex} adds a tiny perturbation to a cat image and generates an adversarial example...


  \begin{figure}[t]
    \centering
\includegraphics[width=\linewidth]{goodfelow.pdf}
    \caption{An example of generating adversarial examples. Starting from the clean image on the left, we add a small perturbation using PGD to obtain an adversarial input... The classifier correctly predicts the clean image as a cat (confidence 94.6\%), while it misclassifies the adversarial example as a dog (confidence 99.8\%). This illustrates that predictions can change drastically even when the input appears almost identical to humans\cite{Goodfellow2014ExplainingAH}.}
    \label{fig:advex}
  \end{figure}


\subsection{Adversarial attack: FGSM and PGD}\label{AA}
We refer to methods for generating adversarial examples as \emph{adversarial attacks}, and many variants have been studied. In this work, we focus on representative first-order attacks: the Fast Gradient Sign Method (FGSM)~\cite{Goodfellow2014ExplainingAH} and Projected Gradient Descent (PGD)~\cite{Madry2017TowardsDL}. Here, $\mathrm{sign}:\mathbb{R} \rightarrow [-1,1]$ is applied element-wise to the argument vector.
\begin{equation}
\bm{x}'=\bm{x}+\epsilon\cdot\mathrm{sign}\left(\nabla_{\bm{x}}
\ell(g(\bm{x};\bm{\theta}),y)\right)
\label{fgsm}
\end{equation}
This produces an input that increases the loss when fed into the model. A stronger iterative variant of this method is PGD~\cite{Madry2017TowardsDL}, which updates the input in the direction that increases the loss with step size $\alpha$, similarly to FGSM, and then applies the projection $\Pi_{\mathcal{...}}$.
In particular,
\[
\mathcal{B}_\infty(\bm{x},\epsilon):=\{\bm{z}\in\mathbb{R}^d \mid \|\bm{z}-\bm{x}\|_\infty \le \epsilon\}
\]
Then, $\Pi_{\mathcal{B}_\infty(\bm{x},\epsilon)}$ denotes the projection onto the $\ell_\infty$-ball, which guarantees $\|\bm{x}'-\bm{x}\|_\infty \le \epsilon$.
Under this setting, the PGD update is given by:
\begin{equation}
\bm{x}^{\prime} \leftarrow \Pi_{\mathcal{B}_\infty(\bm{x},\epsilon)}
\left[\bm{x}+\alpha\,\operatorname{sign}\left(\nabla_{\bm{x}} \ell(g(\bm{x} ; \bm{\theta}),y)\right)\right].
\label{pgd}
\end{equation}


% In particular, $\operatorname{Clip}{(\bm{x}-\epsilon,\ \bm{x}+\epsilo...$ projects by restricting each component to the interval $[\bm{x}-\epsilon,\ \bm{x}+\epsilon]$, ensuring $|\bm{x}'-\bm...$.

By repeating Eq.~\eqref{pgd} multiple times, we can generate samples that more strongly increase the loss within the $\epsilon$-ball.

\subsection{Adversarial Training}\label{AT}

Adversarial training improves model robustness by training on adversarial examples~\cite{Madry2017TowardsDL}. It is typically formulated as the following min--max optimization problem: one first generates adversarial examples for each input, and then learns parameters that minimize the average loss over these adversarial inputs.
\begin{equation}
\min_{\boldsymbol{\theta}} \frac{1}{n} \sum_{i=1}^n
\max_{\|\bm{x}_i^{\prime}-\bm{x}_i\|_\infty \le \epsilon}
\ell\!\left(g(\bm{x}_i^{\prime};\boldsymbol{\theta}), y_i\right).
\end{equation}
In addition, as a method to further enhance robustness against adversarial examples,
TRADES (TRadeoff-inspired Adversarial DEfense via Surrogate-loss minimization), proposed by Zhang \emph{et al.}\cite{zhang2019theoretically}, is a prominent approach.
TRADES explicitly models the trade-off between accuracy on clean samples and robustness to adversarial samples,
and aims to minimize the following loss function:
\begin{equation}
\begin{aligned}
\mathcal{L}_{\mathrm{TR}}(\boldsymbol{\theta})
    &= \frac{1}{n} \sum_{i=1}^n \Bigl[
        \ell\bigl(g(\bm{x}_i ; \boldsymbol{\theta}), y_i \bigr)
        \\&\quad
        + \beta \cdot \max_{\|\bm{x}_i^{\prime} - \bm{x}_i\|_\infty \le \epsilon}
        \ell_{\mathrm{KL}}\bigl(g(\bm{x}_i ; \boldsymbol{\theta}), g(\bm{x}_i^{\prime} ; \boldsymbol{\theta}) \bigr)
    \Bigr].
\end{aligned}
\label{eq:trades_loss}
\end{equation}
Here, $\ell_{\mathrm{KL}}(\cdot,\cdot)$ denotes the Kullback--Leibler (KL) divergence between predictive distributions, i.e.,
\[
\ell_{\mathrm{KL}}\bigl(g(\bm{x}_i;\boldsymbol{\theta}),\,g(\bm{x}_i';\boldsymbol{\theta})\bigr)
:= \mathrm{KL}\!\left(
p_{\boldsymbol{\theta}}(\cdot\mid \bm{x}_i)\ \big\|\ p_{\boldsymbol{\theta}}(\cdot\mid \bm{x}_i')
\right).
\]


The first term of Eq.~\eqref{eq:trades_loss},
$\ell\bigl(g(\bm{x}_i ; \boldsymbol{\theta}), y_i \bigr)$, is
the standard classification loss on the clean input $\bm{x}_i$.

On the other hand, the second term
$\ell_{\mathrm{KL}}\bigl(g(\bm{x}_i ; \boldsymbol{\theta}), g(\bm{x}_i^{\prime} ; \boldsymbol{\theta})\bigr)$
constrains the model so that the output distributions for $\bm{x}_i$ and its perturbed version $\bm{x}_i^{\prime}$ are close,
and this term plays a key role in improving robustness.
Thus, TRADES is designed to enhance robustness while maintaining classification accuracy.
In this study, we apply this framework to PU learning
to achieve both high performance on clean samples and robustness to adversarial examples.





\section{Accurate and Robust PU Learning}\label{ch:method}

% ========================================================================

In this section, we first clarify the issues that arise when uPU learning is naively combined with PGD-based adversarial training.
We then propose a new learning method, PU+TRADES, which adapts the TRADES framework to PU learning.

\subsection{uPU+PGD}

In uPU learning, the empirical risk $\widehat{R}_{\mathrm{uPU}}(g)$ is estimated by minimizing
\begin{align}
\widehat{R}_{\mathrm{uPU}}(g)
= \frac{\pi_{\mathrm{P}}}{n_{\mathrm{P}}}
\sum_{i=1}^{n_{\mathrm{P}}}
\tilde{\ell}\left(g\bigl(\boldsymbol{x}_i^{\mathrm{P}}\bigr),+1\right)
+ \frac{1}{n_{\mathrm{U}}}
\sum_{i=1}^{n_{\mathrm{U}}}
\ell\left(g\bigl(\boldsymbol{x}_i^{\mathrm{U}}\bigr),-1\right).
\end{align}
Here, the loss is taken differently for P and U samples, which can be summarized as
\begin{align}
\mathscr{L}(\boldsymbol{x}):=
\begin{cases}
\tilde{\ell}(g(\boldsymbol{x}), +1), & \boldsymbol{x} \in \mathscr{X}_{\mathrm{P}},\\[2pt]
\ell(g(\boldsymbol{x}), -1),         & \boldsymbol{x} \in \mathscr{X}_{\mathrm{U}}.
\end{cases}
\label{eq:uPU_loss_piecewise}
\end{align}
Using this loss $\mathscr{L}$, we generate an adversarial example for each sample via PGD.
A single PGD update step is given by
\begin{align}
\boldsymbol{x}^{\prime} \leftarrow
\operatorname{Clip}_{(\boldsymbol{x}-\epsilon,\ \boldsymbol{x}+\epsilon)}
\Bigl[\boldsymbol{x}^{\prime}+\alpha\,\operatorname{sign}\!\bigl(\nabla_{\boldsymbol{x}^{\prime}} \mathscr{L}(\boldsymbol{x}^{\prime})\bigr)\Bigr].
\end{align}

\subsection*{Issues with uPU+PGD}

In uPU, the loss for unlabeled data is computed as if the label were always $y=-1$.
However, in reality, the unlabeled set contains a mixture of positives and negatives.
This property is incompatible with PGD-based adversarial training.

\begin{itemize}
  \item \textbf{Negative U samples.}
  Since the loss $\ell(g(\boldsymbol{x}^{\mathrm{U}}), -1)$ is consistent with the true label, it pushes the input in a direction that increases the loss for the negative class.
  Consequently, PGD generates appropriate adversarial perturbations, contributing to improved robustness.

  \item \textbf{Positive U samples.}
  If perturbations are generated using $\ell(g(\boldsymbol{x}^{\mathrm{U}}), -1)$ even though the sample is truly positive, PGD updates the input so as to maximize the \emph{negative-class} loss.
  As a result, the input may be pushed not toward the decision boundary, but rather toward a region where it is classified as positive with higher confidence.
  Therefore, PGD fails to produce perturbations in the “most misclassifiable direction,” and the training can break down.
\end{itemize}

Hence, to generate adversarial perturbations appropriately in PU learning, it is essential to use a \emph{label-independent} perturbation generation mechanism.
This motivates PU+TRADES, introduced in the next section.

% ----------------------------------------------------------

\subsection{PU+TRADES}

In this work, we propose \textbf{uPU+TRADES} and \textbf{nnPU+TRADES}, which integrate TRADES into uPU and nnPU, respectively.
By introducing the TRADES framework into PU learning, we endow the model with robustness.

The objective function of uPU+TRADES is given by
\begin{equation}
\min_{g}\left[
\widehat{R}_{\mathrm{uPU}}(g)
+\beta \cdot \frac{1}{n} \sum_{i=1}^n
\max_{\left\|\boldsymbol{x}^{\prime}-\boldsymbol{x}_i\right\|_{\infty} \leq \epsilon}
\ell_{\mathrm{KL}}\left(g\left(\boldsymbol{x}_i\right)\ \|\ g\left(\boldsymbol{x}^{\prime}\right)\right)
\right],
\end{equation}
and the objective function of nnPU+TRADES is given by
\begin{equation}
\min_{g}\left[
\widehat{R}_{\mathrm{nnPU}}(g)
+\beta \cdot \frac{1}{n} \sum_{i=1}^n
\max_{\left\|\boldsymbol{x}^{\prime}-\boldsymbol{x}_i\right\|_{\infty} \leq \epsilon}
\ell_{\mathrm{KL}}\left(g\left(\boldsymbol{x}_i\right)\ \|\ g\left(\boldsymbol{x}^{\prime}\right)\right)
\right].
\end{equation}

In binary classification, the network outputs a one-dimensional logit $g(\boldsymbol{x};\boldsymbol{\theta})$.
We convert it into a Bernoulli probability vector and compute the KL divergence:
\begin{align}
p(\boldsymbol{x})=
\bigl[\sigma(g(\boldsymbol{x}; \boldsymbol{\theta})),\, 1-\sigma(g(\boldsymbol{x}; \boldsymbol{\theta}))\bigr],
\end{align}
where $\sigma(\cdot)$ denotes the sigmoid function.
The KL loss is then defined as
\begin{equation}
\begin{aligned}
\ell_{\mathrm{KL}}\left(g\left(\boldsymbol{x}_i; \boldsymbol{\theta}\right), g\left(\boldsymbol{x}_i^{\prime}; \boldsymbol{\theta}\right)\right)
&= \mathrm{KL}\bigl(p(\boldsymbol{x}_i)\,\|\,p(\boldsymbol{x}_i^{\prime})\bigr) \\
&= \sum_{c \in \{0,1\}} p_c(\boldsymbol{x}_i)\,
\log \frac{p_c(\boldsymbol{x}_i)}{p_c(\boldsymbol{x}_i^{\prime})}.
\end{aligned}
\end{equation}

This term encourages the model outputs to remain stable under small perturbations of $\boldsymbol{x}_i$,
thereby providing robustness against adversarial perturbations.






\section{Appendix} \label{sec:appendix}

\subsection{Theoretical Analysis} \label{ch:theory}

In this chapter, we consider a binary classification problem where input data may be subject to adversarial perturbations, and derive upper bounds on the gap between the true adversarial risk and its empirical estimator for both supervised learning and PU learning. By comparing these bounds, we theoretically clarify conditions under which PU learning can be advantageous over supervised learning in the finite-sample regime.

Below we define the problem setting used throughout this chapter.

\begin{problem}{(Adversarial Binary Classification Setting)}
\label{problem_adv}

Let the input space be $\mathcal{X} \subseteq \mathbb{R}^d$,
and the label space be $\mathcal{Y}=\{-1, +1\}$.
Assume the input $\boldsymbol{x} \in \mathcal{X}$ is bounded,
i.e., there exists a constant $C_x>0$ such that
\[
\|\boldsymbol{x}\|_\infty \leq C_x
\]
holds.

Assume $(\boldsymbol{x},y)$ is generated from a joint distribution $p(\boldsymbol{x},y)$, and define the class-conditional distributions as

\[
p_{\mathrm{P}}(\boldsymbol{x}) = p(\boldsymbol{x}\mid y=+1), \quad
p_{\mathrm{N}}(\boldsymbol{x}) = p(\boldsymbol{x}\mid y=-1)
\]
as above.
Let the class prior probabilities be
$\pi_{\mathrm{P}} = p(y=+1)$,
$\pi_{\mathrm{N}} = p(y=-1)$
and let
with $\pi_{\mathrm{P}}+\pi_{\mathrm{N}}=1$.

In supervised learning, we use labeled data from the positive (P) and negative (N) classes.
The corresponding datasets are
\begin{equation}
\begin{aligned}
\mathscr{X}_{\mathrm{P}} &= \{\boldsymbol{x}_i^{\mathrm{P}}\}_{i=1}^{n_{\mathrm{P}}}
\stackrel{\mathrm{i.i.d.}}{\sim} p_{\mathrm{P}}(\boldsymbol{x}),\\
\mathscr{X}_{\mathrm{N}} &= \{\boldsymbol{x}_i^{\mathrm{N}}\}_{i=1}^{n_{\mathrm{N}}}
\stackrel{\mathrm{i.i.d.}}{\sim} p_{\mathrm{N}}(\boldsymbol{x})
\end{aligned}
\end{equation}
given by

In PU learning, we use positive data and unlabeled (U) data.
The marginal distribution of unlabeled data is
\[
p_{\mathrm{U}}(\boldsymbol{x})
= \pi_{\mathrm{P}} p_{\mathrm{P}}(\boldsymbol{x})
+ \pi_{\mathrm{N}} p_{\mathrm{N}}(\boldsymbol{x})
\]
given by
and the unlabeled dataset is
\begin{equation}
\mathscr{X}_{\mathrm{U}}
= \{\boldsymbol{x}_i^{\mathrm{U}}\}_{i=1}^{n_{\mathrm{U}}}
\stackrel{\mathrm{i.i.d.}}{\sim} p_{\mathrm{U}}(\boldsymbol{x})
\end{equation}
 .

Let the classifier be $g:\mathbb{R}^d \rightarrow \mathbb{R}$, and
consider a linear classifier parameterized by a weight vector $\boldsymbol{w}\in\mathbb{R}^d$:
\[
g(\boldsymbol{x}) = \boldsymbol{w}^\top \boldsymbol{x}
\]

For $p\ge1$ and $W>0$, define the hypothesis class as
\[
\mathscr{G}
= \{ g(\boldsymbol{x}): \|\boldsymbol{w}\|_p \le W \}
\]
as above.

Let the loss function $\ell:\mathbb{R}\times\mathcal{Y}\rightarrow\mathbb{R}$ take as arguments the classifier output and the true label.

Moreover, as the adversarial regularization term in TRADES we use
\[
\ell_{\mathrm{KL}}\big(g(\bm x),g(\bm x')\big)
\]
where $\ell_{\mathrm{KL}}$ denotes
the Kullback--Leibler divergence between probability distributions induced by the classifier outputs.

Throughout this chapter, we assume that $\ell$ and $\ell_{\mathrm{KL}}$ satisfy the regularity conditions needed for the analysis (boundedness and Lipschitz continuity); see the assumptions in Theorem\ref{thm:pntr-estimation} for details.
\end{problem}


\subsection{Preliminaries (Notation and Assumptions)}\label{subsec:theory_tools}

To derive the estimation-error upper bounds in this chapter, we use the Rademacher complexity to bound the expected uniform deviation,
and we use McDiarmid's inequality to obtain high-probability guarantees that hold with probability at least $1-\delta$.
Below we summarize the definitions and properties used in this chapter.

\paragraph{Rademacher Complexity}\label{rad}
In our discussion, we need a measure of the complexity of a function class $\mathscr{G}$.
We adopt the Rademacher complexity and recall its definition.
A random variable $\sigma$ satisfying $\Pr(\sigma=+1)=\Pr(\sigma=-1)=1/2$ is called a Rademacher variable.
Given a set of $n$-dimensional vectors $S\subseteq\mathbb{R}^n$,
the Rademacher complexity of $S$ is defined as
\[
\mathfrak{R}(S)
=\mathbb{E}_{\sigma_1,\ldots,\sigma_n}\left[
\sup_{(s_1,\ldots,s_n)\in S}\frac{1}{n}\sum_{i=1}^n \sigma_i s_i
\right]
\]
as above.
It can be interpreted as the expected correlation between random noise and the best-matching element of $S$.

Next, let $S_n=\{\bm{x}_i\}_{i=1}^n$ be a dataset of size $n$.
For a function class $\mathscr{G}$, define
\[
\mathscr{G}\circ S_n
=\left\{\left(g(\bm{x}_1),\ldots,g(\bm{x}_n)\right)\ \middle|\ g\in\mathscr{G}\right\}
\]
Then,
the empirical Rademacher complexity of $\mathscr{G}$ on $S_n$ is
\[
\mathfrak{R}_{S_n}(\mathscr{G})
=\mathfrak{R}(\mathscr{G}\circ S_n)
=\mathbb{E}_{\sigma_1,\ldots,\sigma_n}\left[
\sup_{g\in\mathscr{G}}\frac{1}{n}\sum_{i=1}^n \sigma_i g(\bm{x}_i)
\right]
\]
given by
Furthermore, when $S_n=\{\bm{x}_i\}_{i=1}^n \stackrel{\text{i.i.d.}}{\sim} \nu(\bm{x})$, we define the Rademacher complexity of $\mathscr{G}$ as follows (where $\nu(\bm{x})$ denotes the distribution of $\bm{x}$).

\begin{tcolorbox}[
    colback = white,
    colframe = black!60,
    boxrule = 0.35pt,
    fonttitle = \bfseries,
    breakable = false]
\begin{dfn}[Rademacher Complexity]\label{thm:def_rademacher}
Let $n$ be the sample size and let $S_n=\{\bm{x}_i\}_{i=1}^n \stackrel{\text{i.i.d.}}{\sim} \nu(\bm{x})$.
Then the Rademacher complexity of $\mathscr{G}$ is
\begin{equation}
 \mathfrak{R}_{n, \nu}(\mathscr{G})
 =\mathbb{E}_{S_n \sim \nu^n}\left[\mathfrak{R}_{S_n}(\mathscr{G})\right]
 =\mathbb{E}_{\bm{x}_1,\ldots,\bm{x}_n}
 \mathbb{E}_{\sigma_1,\ldots,\sigma_n}\left[
 \sup_{g \in \mathscr{G}} \frac{1}{n} \sum_{i=1}^n \sigma_i g(\bm{x}_i)
 \right]
\label{eq:emprad}
\end{equation}
is defined as above.
\end{dfn}
\end{tcolorbox}

\par\noindent
Standard results used in this chapter (Talagrand's contraction lemma, vector contraction, Rademacher complexity bounds for linear function classes, additional bounds for adversarial inputs, McDiarmid's inequality, etc.) are summarized in Appendix\ref{app:toolbox}.

\par\noindent\textit{(Proofs of theorems and lemmas are provided in Appendix~\ref{app:theory-proofs}.)}

\subsection{Upper Bound on the Estimation Error of Supervised TRADES}

In this section, we derive an upper bound on the estimation error of supervised TRADES.
The proof proceeds by (i) deriving an upper bound on the uniform deviation (lemma),
and (ii) applying the standard ERM argument to obtain the estimation-error bound (theorem),
following the standard flow.

\paragraph{Supervised TRADES risk (population risk and empirical risk)}
We define the population risk of supervised TRADES as
\begin{equation}
\begin{aligned}
R_{\mathrm{PN\mbox{-}TR}}(g)
:=
&\ \pi_{\mathrm P}\mathbb{E}_{\mathrm P}\!\left[
\ell\big(g(\bm x),+1\big)+\beta\max_{\|\bm\eta\|_\infty\le\epsilon}
\ell_{\mathrm{KL}}\big(g(\bm x),g(\bm x+\bm\eta)\big)
\right]
\\
&+\pi_{\mathrm N}\mathbb{E}_{\mathrm N}\!\left[
\ell\big(g(\bm x),-1\big)+\beta\max_{\|\bm\eta\|_\infty\le\epsilon}
\ell_{\mathrm{KL}}\big(g(\bm x),g(\bm x+\bm\eta)\big)
\right]
\label{eq:pntr-risk-jp}
\end{aligned}
\end{equation}
as above.
The corresponding empirical risk is
\begin{equation}
\begin{aligned}
\widehat R_{\mathrm{PN\mbox{-}TR}}(g)
:=
&\ \frac{\pi_{\mathrm P}}{n_{\mathrm P}}\sum_{i=1}^{n_{\mathrm P}}
\left[
\ell\big(g(\bm x_i^{\mathrm P}),+1\big)+\beta\max_{\|\bm\eta\|_\infty\le\epsilon}
\ell_{\mathrm{KL}}\big(g(\bm x_i^{\mathrm P}),g(\bm x_i^{\mathrm P}+\bm\eta)\big)
\right]
\\
&+\frac{\pi_{\mathrm N}}{n_{\mathrm N}}\sum_{i=1}^{n_{\mathrm N}}
\left[
\ell\big(g(\bm x_i^{\mathrm N}),-1\big)+\beta\max_{\|\bm\eta\|_\infty\le\epsilon}
\ell_{\mathrm{KL}}\big(g(\bm x_i^{\mathrm N}),g(\bm x_i^{\mathrm N}+\bm\eta)\big)
\right]
\label{eq:pntr-emp-jp}
\end{aligned}
\end{equation}
 .
Define the empirical risk minimizer as
\[
\widehat g_{\mathrm{PN\mbox{-}TR}}
:=\arg\min_{g\in\mathscr G}\widehat R_{\mathrm{PN\mbox{-}TR}}(g)
\]
Also, let
\[
g^*\in\arg\min_{g\in\mathscr G} R_{\mathrm{PN\mbox{-}TR}}(g)
\]
be a population risk minimizer.
With these definitions, we obtain the following upper bound on the estimation error of supervised TRADES.




\begin{tcolorbox}[
    colback = white,
    colframe = black,
    fonttitle = \bfseries,
    breakable = false]
\begin{thm}[Upper Bound on the Estimation Error of Supervised TRADES]
\label{thm:pntr-estimation}

Let a function class $\mathscr{G}$ be given.
Assume the following:

\begin{itemize}
\item \textbf{(Boundedness of the loss)}
One of the following holds:
\begin{itemize}
\item There exists a constant $C_\ell>0$ such that
for any $\widehat y\in\mathbb{R}$ and $y\in\mathcal{Y}$,
$\ell(\widehat y,y)\le C_\ell$ holds; or
\item there exists a constant $C_g>0$ such that
$\|g\|_\infty=\sup_{\bm x\in\mathcal X}|g(\bm x)|\le C_g$
holds for any $g\in\mathscr{G}$,
and for $|\widehat y|\le C_g$,
$\ell(\widehat y,y)\le C_\ell$ holds.
\end{itemize}

\item \textbf{(Lipschitz continuity of the loss)}
$\ell(\widehat y,y)$ is
$L_\ell$-Lipschitz continuous with respect to $\widehat y$.

\item \textbf{(Regularity of the KL term)}
The TRADES regularization term $\ell_{\mathrm{KL}}\left(g(\boldsymbol{x}), g\left(\boldsymbol{x}^{\prime}\right)\right)$
is $L_{\mathrm{KL}}$-Lipschitz continuous in each argument,
and is uniformly bounded:
$\ell_{\mathrm{KL}}(u,v)\le C_{\mathrm{KL}}$
holds.
\end{itemize}

Then, for any $\delta>0$,
the following holds with probability at least $1-\delta$:
\small
\begin{equation}
\begin{aligned}
R_{\mathrm{PN\mbox{-}TR}}(\widehat g_{\mathrm{PN\mbox{-}TR}})
-
R_{\mathrm{PN\mbox{-}TR}}(g^*)
\;\le\;
&
4\big(L_\ell+4\beta L_{\mathrm{KL}}\big)
\Big(
\pi_{\mathrm P}\mathfrak{R}_{n_{\mathrm P},p_{\mathrm P}}(\mathscr{G})
+
\pi_{\mathrm N}\mathfrak{R}_{n_{\mathrm N},p_{\mathrm N}}(\mathscr{G})
\Big)
\\
&+
8\beta L_{\mathrm{KL}}\,\varepsilon W d^{1/q}\left(\frac{\pi_{\mathrm P}}{\sqrt{n_{\mathrm P}}}+\frac{\pi_{\mathrm N}}{\sqrt{n_{\mathrm N}}}\right)
\\
&+
\sqrt{2\ln\frac{2}{\delta}}\,
(C_\ell+\beta C_{\mathrm{KL}})
\left(
\frac{\pi_{\mathrm P}}{\sqrt{n_{\mathrm P}}}
+
\frac{\pi_{\mathrm N}}{\sqrt{n_{\mathrm N}}}
\right).
\end{aligned}
\end{equation}
\end{thm}
\end{tcolorbox}

\paragraph{\textbf{Interpretation and implications (statistical convergence rate)}}
Under the linear-in-parameters model ($\|\bm w\|_p\le W$) and bounded inputs, standard bounds in Appendix~\ref{app:toolbox} yield
\[
\mathfrak{R}_{n_{\mathrm P}, p_{\mathrm P}}(\mathscr G)=\mathcal{O}\!\left(\frac{W}{\sqrt{n_{\mathrm P}}}\right),\qquad
\mathfrak{R}_{n_{\mathrm N}, p_{\mathrm N}}(\mathscr G)=\mathcal{O}\!\left(\frac{W}{\sqrt{n_{\mathrm N}}}\right)
\]
Substituting this bound into Theorem~\ref{thm:pntr-estimation} and absorbing constants such as $\beta$ and $\varepsilon$, the estimation-error upper bound becomes
\[
R_{\mathrm{PN\mbox{-}TR}}(\widehat g_{\mathrm{PN\mbox{-}TR}})-R_{\mathrm{PN\mbox{-}TR}}(g^*)
=
\mathcal{O}_p\!\left(\frac{\pi_{\mathrm P}}{\sqrt{n_{\mathrm P}}}+\frac{\pi_{\mathrm N}}{\sqrt{n_{\mathrm N}}}\right)
\]
This implies an $\mathcal{O}_p\!\left(\pi_{\mathrm P}/\sqrt{n_{\mathrm P}}+\pi_{\mathrm N}/\sqrt{n_{\mathrm N}}\right)$ rate, and in particular the bound converges to $0$ in probability as $n_{\mathrm P},n_{\mathrm N}\to\infty$.

As preparation for Theorem~\ref{thm:pntr-estimation}, we first bound the uniform deviation for supervised TRADES.


\paragraph{Auxiliary lemmas}
In what follows, to bound the Rademacher complexity terms arising from the adversarial component of TRADES,
we use two auxiliary lemmas: Lemma~\ref{lem:advRad} and Lemma~\ref{lem:vecContract}.
(Hereafter, $q$ denotes the conjugate exponent of $p$ (i.e., $1/p+1/q=1$).)




% \begin{proof}
% This can be shown by applying $\ell(u,v)-\ell(u',v')\le L_\ell|u-u'|+L_\ell|v-v'|$ to the Rademacher sum.
% \end{proof}

\begin{tcolorbox}[
    colback = white,
    colframe = black!60,
    boxrule = 0.35pt,
    fonttitle = \bfseries,
    breakable = false]
\begin{lem}[Upper Bound on the Uniform Deviation for Supervised TRADES]
\label{lem:pntr-unif-jp}
Under the above assumptions, for any $\delta>0$, with probability at least $1-\delta$,
\begin{equation}
\begin{aligned}
\sup_{g\in\mathscr G}\left|
\widehat R_{\mathrm{PN\mbox{-}TR}}(g)-R_{\mathrm{PN\mbox{-}TR}}(g)
\right|
\le\;
&
2\bigl(L_\ell+4\beta L_{\mathrm{KL}}\bigr)\Big(\pi_{\mathrm P}\mathfrak{R}_{n_{\mathrm P},p_{\mathrm P}}(\mathscr G)
+
\pi_{\mathrm N}\mathfrak{R}_{n_{\mathrm N},p_{\mathrm N}}(\mathscr G)\Big)
\\
&+
4\beta L_{\mathrm{KL}}\,\varepsilon W d^{1/q}\left(\frac{\pi_{\mathrm P}}{\sqrt{n_{\mathrm P}}}+\frac{\pi_{\mathrm N}}{\sqrt{n_{\mathrm N}}}\right)
\\
&+
\sqrt{\frac{1}{2}\ln\frac{2}{\delta}}\,(C_\ell+\beta C_{\mathrm{KL}})
\left(\frac{\pi_{\mathrm P}}{\sqrt{n_{\mathrm P}}}+\frac{\pi_{\mathrm N}}{\sqrt{n_{\mathrm N}}}\right).
\end{aligned}
\label{eq:pntr-unif-bound-jp}
\end{equation}
holds.
\end{lem}
\end{tcolorbox}


\par\noindent\textit{(Proof is given in Appendix~\ref{app:proof-lem-pntr-unif-jp}.)}




\par\noindent\textit{(Proof is given in Appendix~\ref{app:proof-thm-pntr-estimation}.)}


%============================================================
% uPU+TRADES
%============================================================



\subsection{Upper Bound on the Estimation Error of uPU+TRADES}

In this section, we consider the uPU+TRADES objective based on positive (P) and unlabeled (U) data,
and derive an upper bound on its estimation error.

\paragraph{uPU+TRADES risk (population risk and empirical risk)}
Using the composite loss $\tilde{\ell}(g(\boldsymbol{x}), y)=\ell(g(\boldsymbol{x}), y)-\ell(g(\boldsymbol{x}),-y)$
we define the population risk of uPU+TRADES as
\begin{align}
R_{\mathrm{uPU\mbox{-}TR}}(g)
:=
&\ \pi_{\mathrm P}\mathbb{E}_{\mathrm P}\!\left[
\tilde{\ell}(g(\boldsymbol{x}), +1)
+\beta\max_{\|\bm\eta\|_\infty\le\epsilon}
\ell_{\mathrm{KL}}\big(g(\bm x+\bm\eta),g(\bm x)\big)
\right]
\nonumber\\
&+\mathbb{E}_{\mathrm U}\!\left[
\ell\big(g(\bm x),-1\big)+\beta\max_{\|\bm\eta\|_\infty\le\epsilon}
\ell_{\mathrm{KL}}\big(g(\bm x+\bm\eta),g(\bm x)\big)
\right]
\label{eq:uputr-risk-jp}
\end{align}
as above.
The corresponding empirical risk is
\begin{align}
\widehat R_{\mathrm{uPU\mbox{-}TR}}(g)
:=
&\ \frac{\pi_{\mathrm P}}{n_{\mathrm P}}\sum_{i=1}^{n_{\mathrm P}}
\left[
\tilde{\ell}(g(\boldsymbol{x}_i^{\mathrm P}), +1)
+\beta\max_{\|\bm\eta\|_\infty\le\epsilon}
\ell_{\mathrm{KL}}\big(g(\bm x_i^{\mathrm P}+\bm\eta),g(\bm x_i^{\mathrm P})\big)
\right]
\nonumber\\
&+\frac{1}{n_{\mathrm U}}\sum_{i=1}^{n_{\mathrm U}}
\left[
\ell\big(g(\bm x_i^{\mathrm U}),-1\big)
+\beta\max_{\|\bm\eta\|_\infty\le\epsilon}
\ell_{\mathrm{KL}}\big(g(\bm x_i^{\mathrm U}+\bm\eta),g(\bm x_i^{\mathrm U})\big)
\right]
\label{eq:uputr-emp-jp}
\end{align}
 .
Define the empirical risk minimizer as
\[
\widehat g_{\mathrm{uPU\mbox{-}TR}}
:=\arg\min_{g\in\mathscr G}\widehat R_{\mathrm{uPU\mbox{-}TR}}(g)
\]
Also, let
\[
g^*\in\arg\min_{g\in\mathscr G} R_{\mathrm{uPU\mbox{-}TR}}(g)
\]
be a population risk minimizer.


\begin{tcolorbox}[
    colback = white,
    colframe = black,
    fonttitle = \bfseries,
    breakable = false,
  %   left=0mm,right=0mm,  % remove left/right padding
  % top=1mm,bottom=1mm
  ]
\begin{thm}[Upper Bound on the Estimation Error of uPU+TRADES]
\label{thm:uputr-estimation}

Let a function class $\mathscr{G}$ be given.
Assume the following (as in the previous subsection):

\begin{itemize}
\item \textbf{(Boundedness of the loss)} There exist constants $C_\ell, C_{\mathrm{KL}}>0$ such that
for any $y\in\mathcal{Y}$ and any input, the following holds:
\begin{itemize}
\item (Classification loss) One of the following holds:
\begin{itemize}
\item There exists a constant $C_\ell>0$ such that for any $\widehat y\in\mathbb{R}$,
$\ell(\widehat y,y)\le C_\ell$, or
\item there exists a constant $C_g>0$ such that $\|g\|_\infty\le C_g$ (for $g\in\mathscr{G}$) and
for $|\widehat y|\le C_g$, $\ell(\widehat y,y)\le C_\ell$.
\end{itemize}
\item (TRADES term) For any $u,v$,
$\ell_{\mathrm{KL}}(u,v)\le C_{\mathrm{KL}}$.
\end{itemize}

\item \textbf{(Lipschitz continuity)} There exist constants $L_\ell, L_{\mathrm{KL}}>0$ such that
\begin{itemize}
\item (Classification loss) $\ell(\widehat y,y)$ is $L_\ell$-Lipschitz continuous with respect to $\widehat y$.
\item (TRADES term) $\ell_{\mathrm{KL}}(u,v)$ is $L_{\mathrm{KL}}$-Lipschitz continuous in each argument
\end{itemize}
\end{itemize}

Then, for any $\delta>0$,
the following holds with probability at least $1-\delta$:
\footnotesize
\begin{align}
R_{\mathrm{uPU\mbox{-}TR}}(\widehat g_{\mathrm{uPU\mbox{-}TR}})-R_{\mathrm{uPU\mbox{-}TR}}(g^*)
\ \le\
&
8\pi_{\mathrm P}\big(L_\ell+2\beta L_{\mathrm{KL}}\big)\mathfrak{R}_{n_{\mathrm P},p_{\mathrm P}}(\mathscr G) \nonumber\\
& +
4\big(L_\ell+4\beta L_{\mathrm{KL}}\big)\mathfrak{R}_{n_{\mathrm U},p_{\mathrm U}}(\mathscr G)
\nonumber\\
&+
8\beta L_{\mathrm{KL}}\,\varepsilon W d^{1/q}
\left(
\frac{\pi_{\mathrm P}}{\sqrt{n_{\mathrm P}}}
+\frac{1}{\sqrt{n_{\mathrm U}}}
\right)
\nonumber\\
&+
\sqrt{2\ln\frac{2}{\delta}}
\left(
\frac{\pi_{\mathrm P}(2C_\ell+\beta C_{\mathrm{KL}})}{\sqrt{n_{\mathrm P}}}
+
\frac{C_\ell+\beta C_{\mathrm{KL}}}{\sqrt{n_{\mathrm U}}}
\right).
\label{eq:uputr-estimation-bound}
\end{align}
\end{thm}
\end{tcolorbox}

\paragraph{\textbf{Interpretation and implications (statistical convergence rate)}}
Under the linear-in-parameters model ($\|\bm w\|_p\le W$) and bounded inputs, standard bounds in Appendix~\ref{app:toolbox} yield
\[
\mathfrak{R}_{n_{\mathrm P}, p_{\mathrm P}}(\mathscr G)=\mathcal{O}\!\left(\frac{W}{\sqrt{n_{\mathrm P}}}\right),\qquad
\mathfrak{R}_{n_{\mathrm U}, p_{\mathrm U}}(\mathscr G)=\mathcal{O}\!\left(\frac{W}{\sqrt{n_{\mathrm U}}}\right)
\]
Substituting this bound into \eqref{eq:uputr-estimation-bound} and absorbing constants such as $\beta$ and $\varepsilon$, the estimation-error upper bound becomes
\[
R_{\mathrm{uPU\mbox{-}TR}}(\widehat g_{\mathrm{uPU\mbox{-}TR}})-R_{\mathrm{uPU\mbox{-}TR}}(g^*)
=
\mathcal{O}_p\!\left(\frac{\pi_{\mathrm P}}{\sqrt{n_{\mathrm P}}}+\frac{1}{\sqrt{n_{\mathrm U}}}\right)
\]
This yields an upper bound that converges to $0$ in probability as $n_{\mathrm P},n_{\mathrm U}\to\infty$.

As preparation for Theorem~\ref{thm:uputr-estimation},
we present a lemma bounding the uniform deviation for uPU+TRADES.


\paragraph{Auxiliary lemmas}
Below we use the auxiliary
Lemma~\ref{lem:advRad} (Rademacher increase under adversarial inputs) and
Lemma~\ref{lem:vecContract} (vector contraction).


\begin{tcolorbox}[
    colback = white,
    colframe = black!60,
    boxrule = 0.35pt,
    fonttitle = \bfseries,
    breakable = false]
\begin{lem}[Upper Bound on the Uniform Deviation for uPU+TRADES]
\label{lem:uputr-unif-jp}
For any $\delta>0$, with probability at least $1-\delta$,
\small
\begin{align}
\sup_{g\in\mathscr G}\left|\widehat R_{\mathrm{uPU\mbox{-}TR}}(g)-R_{\mathrm{uPU\mbox{-}TR}}(g)\right|
\ \le\
&
4\pi_{\mathrm P}\big(L_\ell+2\beta L_{\mathrm{KL}}\big)\mathfrak{R}_{n_{\mathrm P},p_{\mathrm P}}(\mathscr G) \nonumber\\
& +
2\big(L_\ell+4\beta L_{\mathrm{KL}}\big)\mathfrak{R}_{n_{\mathrm U},p_{\mathrm U}}(\mathscr G)
\nonumber\\
&+
4\beta L_{\mathrm{KL}}\,\varepsilon W d^{1/q}
\left(
\frac{\pi_{\mathrm P}}{\sqrt{n_{\mathrm P}}}
+\frac{1}{\sqrt{n_{\mathrm U}}}
\right)
\nonumber\\
&+
\sqrt{\frac{1}{2}\ln\frac{2}{\delta}}
\left(
\frac{\pi_{\mathrm P}(2C_\ell+\beta C_{\mathrm{KL}})}{\sqrt{n_{\mathrm P}}}
+
\frac{C_\ell+\beta C_{\mathrm{KL}}}{\sqrt{n_{\mathrm U}}}
\right)
\label{eq:uputr-unif-bound-jp}
\end{align}
\end{lem}
\end{tcolorbox}


\par\noindent\textit{(Proof is given in Appendix~\ref{app:proof-lem-uputr-unif-jp}.)}



\par\noindent\textit{(Proof is given in Appendix~\ref{app:proof-thm-uputr-estimation}.)}



%============================================================
% nnPU+TRADES
%============================================================

\subsection{Upper Bound on the Estimation Error of nnPU+TRADES}
\label{subsec:nnpu-trades-bound}

In this section, we derive an upper bound on the estimation error of nnPU+TRADES.

\paragraph{nnPU+TRADES risk (population risk and empirical risk)}
First, define the adversarial regularization term in TRADES as
\[
\psi(g,\bm x)
:=
\max_{\|\bm\eta\|_{\infty}\le \varepsilon}
\ell_{\mathrm{KL}}\big(g(\bm x+\bm\eta),\,g(\bm x)\big)
\]
 .
We define the population risk of nnPU+TRADES by adding this regularization term to the nnPU (Kiryo et al., 2017) risk estimator:
\begin{align}
R_{\mathrm{nnPU\mbox{-}TR}}(g)
:={}&
\pi_{\mathrm P}\,\mathbb{E}_{\mathrm P}\!\left[\ell\big(g(\bm x),+1\big)+\beta\,\psi(g,\bm x)\right]
\nonumber\\
&+
\max\Bigl\{0,\,-\pi_{\mathrm P}\,\mathbb{E}_{\mathrm P}\!\left[\ell\big(g(\bm x),-1\big)\right]
+\mathbb{E}_{\mathrm U}\!\left[\ell\big(g(\bm x),-1\big)\right]\Bigr\}
\nonumber\\
&+\beta\,\mathbb{E}_{\mathrm U}\!\left[\psi(g,\bm x)\right]
\label{eq:nnputr-true-risk-jp}
\end{align}
as above.
The corresponding empirical risk is
\begin{align}
\widehat R_{\mathrm{nnPU\mbox{-}TR}}(g)
:={}&
\frac{\pi_{\mathrm P}}{n_{\mathrm P}}\sum_{i=1}^{n_{\mathrm P}}
\Bigl[\ell\big(g(\bm x_i^{\mathrm P}),+1\big)+\beta\,\psi(g,\bm x_i^{\mathrm P})\Bigr]
\nonumber\\
&+
\max\Biggl\{0,\,-\frac{\pi_{\mathrm P}}{n_{\mathrm P}}\sum_{i=1}^{n_{\mathrm P}}\ell\big(g(\bm x_i^{\mathrm P}),-1\big)
+\frac{1}{n_{\mathrm U}}\sum_{i=1}^{n_{\mathrm U}}\ell\big(g(\bm x_i^{\mathrm U}),-1\big)\Biggr\}
\nonumber\\
&+\frac{\beta}{n_{\mathrm U}}\sum_{i=1}^{n_{\mathrm U}}\psi(g,\bm x_i^{\mathrm U})
\label{eq:nnputr-emp-risk-jp}
\end{align}
 .
Define the empirical risk minimizer as
\[
\widehat g_{\mathrm{nnPU\mbox{-}TR}}
:=\arg\min_{g\in\mathscr G}\widehat R_{\mathrm{nnPU\mbox{-}TR}}(g)
\]
Also, let
\[
 g^*\in\arg\min_{g\in\mathscr G} R_{\mathrm{nnPU\mbox{-}TR}}(g)
\]
be a population risk minimizer.


\begin{tcolorbox}[
    colback = white,
    colframe = black,
    fonttitle = \bfseries,
    breakable = false]
\begin{thm}[Upper Bound on the Estimation Error of nnPU+TRADES]
\label{thm:nnputr-estimation}

Let a function class $\mathscr{G}$ be given.
Assume the following (as in the previous subsection):

\begin{itemize}
\item \textbf{(Boundedness of the loss)} There exist constants $C_\ell, C_{\mathrm{KL}}>0$ such that
for any $y\in\mathcal{Y}$ and any input, the following holds:
\begin{itemize}
\item (Classification loss) One of the following holds:
\begin{itemize}
\item There exists a constant $C_\ell>0$ such that for any $\widehat y\in\mathbb{R}$,
$\ell(\widehat y,y)\le C_\ell$, or
\item there exists a constant $C_g>0$ such that $\|g\|_\infty\le C_g$ (for $g\in\mathscr{G}$) and
for $|\widehat y|\le C_g$, $\ell(\widehat y,y)\le C_\ell$.
\end{itemize}
\item (TRADES term) For any $u,v$,
$\ell_{\mathrm{KL}}(u,v)\le C_{\mathrm{KL}}$.
\end{itemize}

\item \textbf{(Lipschitz continuity)} There exist constants $L_\ell, L_{\mathrm{KL}}>0$ such that
\begin{itemize}
\item (Classification loss) $\ell(\widehat y,y)$ is $L_\ell$-Lipschitz continuous with respect to $\widehat y$.
\item (TRADES term) $\ell_{\mathrm{KL}}(u,v)$ is $L_{\mathrm{KL}}$-Lipschitz continuous in each argument
\end{itemize}
\end{itemize}

Then, for any $\delta>0$,
the following holds with probability at least $1-\delta$:
\footnotesize
\begin{align}
R_{\mathrm{nnPU\mbox{-}TR}}(\widehat g_{\mathrm{nnPU\mbox{-}TR}})-R_{\mathrm{nnPU\mbox{-}TR}}(g^*)
\ \le\
&
8\pi_{\mathrm P}\big(L_\ell+2\beta L_{\mathrm{KL}}\big)\mathfrak{R}_{n_{\mathrm P},p_{\mathrm P}}(\mathscr G)
\nonumber\\
&+
4\big(L_\ell+4\beta L_{\mathrm{KL}}\big)\mathfrak{R}_{n_{\mathrm U},p_{\mathrm U}}(\mathscr G)
\nonumber\\
&+
8\beta L_{\mathrm{KL}}\,\varepsilon W d^{1/q}
\left(
\frac{\pi_{\mathrm P}}{\sqrt{n_{\mathrm P}}}
+\frac{1}{\sqrt{n_{\mathrm U}}}
\right)
\nonumber\\
&+
\sqrt{2\ln\frac{2}{\delta}}
\left(
\frac{\pi_{\mathrm P}(2C_\ell+\beta C_{\mathrm{KL}})}{\sqrt{n_{\mathrm P}}}
+
\frac{C_\ell+\beta C_{\mathrm{KL}}}{\sqrt{n_{\mathrm U}}}
\right).
\label{eq:nnputr-estimation-bound}
\end{align}
\end{thm}
\end{tcolorbox}

\paragraph{\textbf{Interpretation and implications (statistical convergence rate)}}
Under the linear-in-parameters model ($\|\bm w\|_p\le W$) and bounded inputs, standard bounds in Appendix~\ref{app:toolbox} yield
\[
\mathfrak{R}_{n_{\mathrm P}, p_{\mathrm P}}(\mathscr G)=\mathcal{O}\!\left(\frac{W}{\sqrt{n_{\mathrm P}}}\right),\qquad
\mathfrak{R}_{n_{\mathrm U}, p_{\mathrm U}}(\mathscr G)=\mathcal{O}\!\left(\frac{W}{\sqrt{n_{\mathrm U}}}\right)
\]
Substituting this bound into \eqref{eq:nnputr-estimation-bound} and absorbing constants such as $\beta$ and $\varepsilon$, the estimation-error upper bound becomes
\[
R_{\mathrm{nnPU\mbox{-}TR}}(\widehat g_{\mathrm{nnPU\mbox{-}TR}})-R_{\mathrm{nnPU\mbox{-}TR}}(g^*)
=
\mathcal{O}_p\!\left(\frac{\pi_{\mathrm P}}{\sqrt{n_{\mathrm P}}}+\frac{1}{\sqrt{n_{\mathrm U}}}\right)
\]
This yields an upper bound that converges to $0$ in probability as $n_{\mathrm P},n_{\mathrm U}\to\infty$.

As preparation for Theorem~\ref{thm:nnputr-estimation}, we provide a lemma bounding the uniform deviation for nnPU+TRADES.


\paragraph{Auxiliary lemmas}
Below we use the auxiliaryLemma~\ref{lem:advRad} (Rademacher increase under adversarial inputs) and
Lemma~\ref{lem:vecContract} (vector contraction).


\begin{tcolorbox}[
    colback = white,
    colframe = black!60,
    boxrule = 0.35pt,
    fonttitle = \bfseries,
    breakable = false]
\begin{lem}[Upper Bound on the Uniform Deviation for nnPU+TRADES]
\label{lem:nnputr-unif-jp}
For any $\delta>0$, with probability at least $1-\delta$,
\footnotesize
\begin{align}
\sup_{g\in\mathscr G}\left|\widehat R_{\mathrm{nnPU\mbox{-}TR}}(g)-R_{\mathrm{nnPU\mbox{-}TR}}(g)\right|
\ \le\
&
4\pi_{\mathrm P}\big(L_\ell+2\beta L_{\mathrm{KL}}\big)\mathfrak{R}_{n_{\mathrm P},p_{\mathrm P}}(\mathscr G)
\nonumber\\
&+
2\big(L_\ell+4\beta L_{\mathrm{KL}}\big)\mathfrak{R}_{n_{\mathrm U},p_{\mathrm U}}(\mathscr G)
\nonumber\\
&+
4\beta L_{\mathrm{KL}}\,\varepsilon W d^{1/q}
\left(
\frac{\pi_{\mathrm P}}{\sqrt{n_{\mathrm P}}}
+\frac{1}{\sqrt{n_{\mathrm U}}}
\right)
\nonumber\\
&+
\sqrt{\frac{1}{2}\ln\frac{2}{\delta}}
\left(
\frac{\pi_{\mathrm P}(2C_\ell+\beta C_{\mathrm{KL}})}{\sqrt{n_{\mathrm P}}}
+
\frac{C_\ell+\beta C_{\mathrm{KL}}}{\sqrt{n_{\mathrm U}}}
\right).
\label{eq:nnputr-unif-bound-jp}
\end{align}
\end{lem}
\end{tcolorbox}


\par\noindent\textit{(Proof is given in Appendix~\ref{app:proof-lem-nnputr-unif-jp}.)}




\par\noindent\textit{(Proof is given in Appendix~\ref{app:proof-thm-nnputr-estimation}.)}


\clearpage
\subsection{Sufficient Unlabeled Sample Size for PU+TRADES to Outperform Supervised TRADES}
\label{subsec:nu_condition_trades}

In this section, we compare the estimation-error upper bounds obtained in the previous subsections (Theorems~\ref{thm:pntr-estimation}, \ref{thm:uputr-estimation}, and \ref{thm:nnputr-estimation}) and derive sufficient conditions on the unlabeled sample size $n_{\mathrm U}$ under which the bound for supervised TRADES (PN+TRADES) becomes larger than that for PU+TRADES (uPU+TRADES / nnPU+TRADES), i.e., conditions under which PU+TRADES can theoretically outperform supervised TRADES.

\paragraph{Rademacher complexity of the linear hypothesis class}
Under the assumptions in this chapter ($\|\bm x\|_\infty\le C_x$ and $\|\bm w\|_p\le W$),
for any distribution $\nu$,
\begin{align}
\mathfrak{R}_{n,\nu}(\mathscr G)
=\mathbb E_{\bm x_{1:n}\sim\nu}\,\mathbb E_{\bm\sigma}
\Big[\sup_{\|\bm w\|_p\le W}\frac{1}{n}\sum_{i=1}^n\sigma_i\bm w^\top\bm x_i\Big]
\ \le\ \frac{W\,\sup_{\bm x\in\mathcal X}\|\bm x\|_q}{\sqrt n}
\ \le\ \frac{W C_x\,d^{1/q}}{\sqrt n}
\label{eq:rad-linear-basic}
\end{align}
holds (where $1/p+1/q=1$).
In the following, for notational clarity,
\begin{equation}
\kappa_\delta:=\sqrt{2\ln\frac{2}{\delta}}
\end{equation}
and we rewrite the estimation-error bounds into a $1/\sqrt{n}$ form for comparison.
Moreover, to collect constant factors,
\begin{equation}
\Gamma_\delta
:=
4\big(L_\ell+4\beta L_{\mathrm{KL}}\big)\,W C_x d^{1/q}
+
8\beta L_{\mathrm{KL}}\,\varepsilon W d^{1/q}
+
\kappa_\delta\,(C_\ell+\beta C_{\mathrm{KL}})
\label{eq:Gamma-delta-def}
\end{equation}
 .

\paragraph{(1) Comparing PN+TRADES and uPU+TRADES}
First, applying \eqref{eq:rad-linear-basic} to Theorem~\ref{thm:pntr-estimation} gives, with probability at least $1-\delta$,
\begin{align}
R_{\mathrm{PN\mbox{-}TR}}(\widehat g_{\mathrm{PN\mbox{-}TR}})-R_{\mathrm{PN\mbox{-}TR}}(g^*)
\ \le\
\Gamma_\delta\Big(\frac{\pi_{\mathrm P}}{\sqrt{n_{\mathrm P}}}+\frac{\pi_{\mathrm N}}{\sqrt{n_{\mathrm N}}}\Big)
\label{eq:pntr-bound-simplified}
\end{align}
is obtained.

Similarly, applying \eqref{eq:rad-linear-basic} to Theorem~\ref{thm:uputr-estimation} yields, with probability at least $1-\delta$,
\begin{align}
R_{\mathrm{uPU\mbox{-}TR}}(\widehat g_{\mathrm{uPU\mbox{-}TR}})-R_{\mathrm{uPU\mbox{-}TR}}(g^*)
\ \le\
\frac{\pi_{\mathrm P}}{\sqrt{n_{\mathrm P}}}\Big(\Gamma_\delta+4L_\ell\,W C_x d^{1/q}+\kappa_\delta C_\ell\Big)
+\frac{\Gamma_\delta}{\sqrt{n_{\mathrm U}}}
\label{eq:uputr-bound-simplified}
\end{align}
follows.

\paragraph{Main result: sufficient unlabeled sample size}
The following theorem, based on the comparison of the estimation-error upper bounds,
provides a sufficient condition on
the unlabeled sample size $n_{\mathrm U}$ under which PU+TRADES can outperform supervised TRADES.

\begin{tcolorbox}[
    colback = white,
    colframe = black,
    fonttitle = \bfseries,
    breakable = false]
\begin{thm}[Sufficient Unlabeled Sample Size for PU+TRADES to Outperform Supervised TRADES]
\label{thm:nu-threshold-trades}
We compare the estimation-error upper bounds in Theorems~\ref{thm:pntr-estimation}, \ref{thm:uputr-estimation}, and \ref{thm:nnputr-estimation}.
Using \eqref{eq:rad-linear-basic}, we reduce them to $1/\sqrt{n}$-type forms,
and we use $\Gamma_\delta$ and $\kappa_\delta$ defined in \eqref{eq:Gamma-delta-def}.
Then
\begin{equation}
\Gamma_\delta\frac{\pi_{\mathrm N}}{\sqrt{n_{\mathrm N}}}
\ >\
\frac{\pi_{\mathrm P}}{\sqrt{n_{\mathrm P}}}\Big(4L_\ell\,W C_x d^{1/q}+\kappa_\delta C_\ell\Big)
\label{eq:compare-pn-uputr-feasible}
\end{equation}
assume it holds.

\begin{enumerate}
\item[(i)] (PN+TRADES vs uPU+TRADES)
If
\begin{equation}
\boxed{\
\ n_{\mathrm U}
\ >\
\left(
\frac{\Gamma_\delta}
{\Gamma_\delta\frac{\pi_{\mathrm N}}{\sqrt{n_{\mathrm N}}}
-\frac{\pi_{\mathrm P}}{\sqrt{n_{\mathrm P}}}\big(4L_\ell\,W C_x d^{1/q}+\kappa_\delta C_\ell\big)}
\right)^{\!2}
\ }
\label{eq:nu-threshold-uputr}
\end{equation}
as a comparison of the estimation-error upper bounds,
the bound for PN+TRADES (\eqref{eq:pntr-bound-simplified})
is larger than that for uPU+TRADES (\eqref{eq:uputr-bound-simplified}).

\item[(ii)] (PN+TRADES vs nnPU+TRADES)
Since the right-hand side of Theorem~\ref{thm:nnputr-estimation} is identical to that of Theorem~\ref{thm:uputr-estimation},
the same sufficient condition
\begin{equation}
\boxed{\
\ n_{\mathrm U}
\ >\
\left(
\frac{\Gamma_\delta}
{\Gamma_\delta\frac{\pi_{\mathrm N}}{\sqrt{n_{\mathrm N}}}
-\frac{\pi_{\mathrm P}}{\sqrt{n_{\mathrm P}}}\big(4L_\ell\,W C_x d^{1/q}+\kappa_\delta C_\ell\big)}
\right)^{\!2}
\ }
\label{eq:nu-threshold-nnputr}
\end{equation}
implies that the PN+TRADES estimation-error upper bound is larger than the nnPU+TRADES one.
\end{enumerate}
\end{thm}
\end{tcolorbox}


\par\noindent\textit{(Proof is given in Appendix~\ref{app:proof-thm-nu-threshold-trades}.)}


If condition \eqref{eq:compare-pn-uputr-feasible} does not hold,
the denominator on the right-hand side becomes non-positive, and thus there is no $n_{\mathrm U}$ satisfying the comparison inequality in (i);
within the scope of comparisons based on the bounds derived in this chapter,
we cannot claim that PU+TRADES outperforms supervised TRADES.
On the other hand, when \eqref{eq:compare-pn-uputr-feasible} holds,
by \eqref{eq:nu-threshold-uputr} and \eqref{eq:nu-threshold-nnputr},
in the regime where the unlabeled sample size is sufficiently large,
the superiority of PU+TRADES is theoretically guaranteed.



\chapter{Experiments}\label{ch:exp}

In this chapter, we experimentally evaluate the effectiveness of PU+TRADES (uPU+TRADES and nnPU+TRADES), proposed in Chapter~\ref{ch:method}, on multiple benchmark datasets.
Specifically, we compare our methods with existing approaches using clean accuracy (Clean Accuracy) and adversarial accuracy (Adversarial Accuracy) as evaluation metrics, and also analyze the effect of the TRADES coefficient $\beta$.
Furthermore, we conduct additional experiments to examine how the theoretical threshold on the number of unlabeled samples, derived in Chapter~\ref{ch:theory}, corresponds to empirical observations.

\section{Experiments}
\subsection{Experimental Setup}

% requires: \usepackage{booktabs}
% optional: \usepackage{graphicx}  (if you use resizebox)
\begin{table}[t]
 \centering
  \caption{Experimental settings for each dataset (F-MNIST, CIFAR-10, CIFAR-100, and Alzheimer MRI).}
 \label{tab:exp_settings}
 \vspace{5pt}
 \small
 \setlength{\tabcolsep}{3.5pt} % tighten columns
 \resizebox{\linewidth}{!}{ % <- uncomment if still too wide
 \begin{tabular}{l|cccc}
 \toprule
  \textbf{Item} & \textbf{F-MNIST} & \textbf{CIFAR-10} & \textbf{CIFAR-100} & \textbf{Alzheimer} \\
 \midrule
  Dataset & FashionMNIST & CIFAR-10 & CIFAR-100 & Alzheimer MRI \\
  Positive class & Top (0,2,4,6) & Vehicles (0,1,8,9) & organics & AD patients \\
  \# positive samples $n_p$ & 1{,}000 & 1{,}000 & 1{,}000 & 769 \\
  \# unlabeled samples $n_u$ & 10{,}000 & 10{,}000 & 50{,}000 & 5{,}121 \\
  Class prior $\pi_p$ & 0.4 & 0.4 & 0.5 & 0.5 \\
  Model & 6-layer MLP & ResNet-18 & ResNet-18 & ResNet-50 \\
 \midrule
  % Methods & \multicolumn{4}{c}{uPU, nnPU, uPU-PGD, nnPU-PGD, uPU+TRADES, nnPU+TRADES} \\
  Perturbation budget $\epsilon$ & 0.3 & 0.031 & 0.031 & 0.031 \\
  Step size $\alpha$ & 0.01 & 0.031 & 0.031 & 0.031 \\
  \# PGD iterations & \multicolumn{4}{c}{10 steps} \\
  TRADES coefficient $\beta$ & \multicolumn{4}{c}{6, 12, 18} \\
 \midrule
  Metrics & \multicolumn{4}{c}{Clean Accuracy, Adversarial Accuracy} \\
 \bottomrule
 \end{tabular}
  }
\end{table}

\paragraph{Datasets.}
We conduct evaluation experiments on four benchmark datasets: FashionMNIST (F-MNIST~\cite{Xiao2017FashionMNIST}), CIFAR-10~\cite{krizhevsky2009learning}, CIFAR-100~\cite{krizhevsky2009learning}, and the Alzheimer dataset\footnote{Dubey, S. \textit{Alzheimer's Dataset} (Kaggle).\ \url{https://www.kaggle.com/tourist55/alzheimers-dataset-4-class-of-images}}.
For F-MNIST, we consider clothing-image classification; for CIFAR-10, we focus on identifying vehicle classes; for CIFAR-100, we target \texttt{organics}; and for Alzheimer, we aim at recognizing AD patients.
The definition of the positive class, the number of positive samples $n_{\mathrm{P}}$, the number of unlabeled samples $n_{\mathrm{U}}$, the class prior $\pi_{\mathrm{P}}$, and the model architectures for each dataset are summarized in Table~\ref{tab:exp_settings}.

\paragraph{Evaluation metrics.}
To compare the performance of each method, we report clean accuracy (Clean Accuracy) and adversarial accuracy (Adversarial Accuracy) on the test set.
Accuracy is defined as
\[
\text{Accuracy}=\frac{\text{\# correct predictions}}{\text{\# total samples}}.
\]
Adversarial accuracy is computed as the classification accuracy on adversarial examples generated by PGD; the perturbation budget $\epsilon$, step size $\alpha$, and the number of iterations (10 steps) follow Table~\ref{tab:exp_settings}.

\paragraph{Training details.}
As listed in Table~\ref{tab:exp_settings}, we use a 6-layer MLP~\cite{Goodfellow-et-al-2016} for F-MNIST, ResNet-18~\cite{He2016ResNet} for CIFAR-10/100, and ResNet-50~\cite{He2016ResNet} for Alzheimer.
As baselines, for F-MNIST and CIFAR-10 we use uPU and nnPU, as well as PGD-based adversarial training (uPU-PGD and nnPU-PGD) and TRADES-based adversarial training (uPU+TRADES and nnPU+TRADES).
For CIFAR-100 and Alzheimer, training is unstable due to uPU-specific overfitting; thus, we restrict evaluation to nnPU and nnPU+TRADES.
We set the TRADES coefficient to $\beta\in\{6,12,18\}$, and other adversarial-perturbation settings ($\epsilon$, $\alpha$, and the number of PGD iterations) follow Table~\ref{tab:exp_settings}.

% NOTE: CIFAR-100 / Alzheimer are evaluated only with nnPU variants (see Sec.~\ref{sec:main_results}).

\section{Main Results}\label{sec:main_results}

% ---------------------------------------------------------
% Main results (FMNIST / CIFAR-10)
% ---------------------------------------------------------
\begin{table}[t]
 \centering
    \caption{Main results on FashionMNIST (6-layer MLP) and CIFAR-10 (ResNet-18).
We compare uPU/nnPU and their adversarially trained variants (PGD training and TRADES training), and report clean accuracy (Clean) and adversarial accuracy (Adv.) under PGD attacks ($\epsilon$ follows Table~\ref{tab:exp_settings}, 10 steps) on the test set.
For each setting, we select the epoch that achieves the highest Clean Accuracy.
While uPU/nnPU without adversarial training achieve almost zero Adv., TRADES improves robustness without severely degrading clean accuracy.}
 \label{tab:main_results_fm_c10}
 \vspace{5pt}
 \small
 \setlength{\tabcolsep}{4.2pt}
 \begin{tabular}{l|cc|cc}
 \toprule
  \multirow{2}{*}{Method} & \multicolumn{2}{c|}{F-MNIST} & \multicolumn{2}{c}{CIFAR-10} \\
 \cline{2-5}
  & Clean & Adv. & Clean & Adv. \\
 \midrule
  uPU  & 0.937 & 0.190 & 0.850 & 0.114 \\
  nnPU  & \textbf{0.948} & 0.001 & \textbf{0.887} & 0.001 \\
  uPU-PGD  & 0.935  & 0.843  & 0.802  & 0.714  \\
  nnPU-PGD  & 0.930 & 0.860 & 0.732 & 0.686 \\
  uPU+TRADES  & 0.934 & 0.914 & 0.845 & 0.711 \\
  nnPU+TRADES & 0.944 & \textbf{0.928} & 0.850 & \textbf{0.723} \\
 \bottomrule
 \end{tabular}
\end{table}

% ---------------------------------------------------------
% Main results (CIFAR-100 / Alzheimer): nnPU variants only
% ---------------------------------------------------------
\begin{table}[t]
 \centering
    \caption{Main results on CIFAR-100 (ResNet-18) and Alzheimer MRI (ResNet-50) (nnPU variants only).
We compare clean accuracy (Clean) and adversarial accuracy (Adv.) under PGD attacks ($\epsilon$ follows Table~\ref{tab:exp_settings}, 10 steps) on the test set.
For each setting, we select the epoch that achieves the highest Clean Accuracy.
While nnPU alone attains almost zero Adv., nnPU+TRADES substantially improves robustness.}
 \label{tab:main_results_c100_ad}
 \vspace{5pt}
 \small
 \setlength{\tabcolsep}{4.2pt}
 \begin{tabular}{l|cc|cc}
 \toprule
  \multirow{2}{*}{Method} & \multicolumn{2}{c|}{CIFAR-100} & \multicolumn{2}{c}{Alzheimer} \\
 \cline{2-5}
  & Clean & Adv. & Clean & Adv. \\
 \midrule
  nnPU  & \textbf{0.680} & 0.001 & \textbf{0.683} & 0.000 \\
  nnPU+TRADES & 0.645 & \textbf{0.450} & 0.649 & \textbf{0.409} \\
 \bottomrule
 \end{tabular}
\end{table}

Tables~\ref{tab:main_results_fm_c10} and \ref{tab:main_results_c100_ad} report clean accuracy (Clean) and adversarial accuracy (Adv.) on each dataset.
First, while uPU and nnPU achieve high clean accuracy, their adversarial accuracy is extremely low, indicating vulnerability to perturbations.
In contrast, nnPU-PGD and (uPU/nnPU)+TRADES markedly improve adversarial accuracy, confirming gains in robustness.

Moreover, on F-MNIST and CIFAR-10, nnPU+TRADES consistently achieves higher adversarial accuracy than uPU+TRADES.
On the other hand, methods with adversarial training tend to suffer a drop in clean accuracy, revealing a trade-off between accuracy and robustness.
For CIFAR-100 and Alzheimer (Table~\ref{tab:main_results_c100_ad}), nnPU+TRADES maintains non-trivial adversarial accuracy while retaining reasonable clean accuracy on both datasets.

\begin{table}[t]
 \centering
    \caption{Ablation on the TRADES coefficient $\beta$ (F-MNIST / CIFAR-10).
We vary $\beta\in\{6,12,18\}$ and compare clean accuracy (Clean Acc.) and adversarial accuracy (Adv. Acc.) for uPU+TRADES and nnPU+TRADES.
On FashionMNIST, performance changes only mildly with $\beta$, whereas on CIFAR-10 increasing $\beta$ tends to improve adversarial accuracy at the cost of decreased clean accuracy.}
 \label{tab:beta_sweep}
 \vspace{5pt}
 \small
 \setlength{\tabcolsep}{4.0pt}
 \begin{tabular}{l|cc|cc}
 \toprule
  \multirow{2}{*}{Method} &
  \multicolumn{2}{c|}{F-MNIST} &
 \multicolumn{2}{c}{CIFAR-10} \\
 \cline{2-5}
  & Clean Acc. & Adv. Acc. & Clean Acc. & Adv. Acc. \\
 \midrule
  uPU+TRADES ($\beta=6$)  & 0.939 & 0.861 & 0.846 & 0.711 \\
  uPU+TRADES ($\beta=12$)  & 0.935 & 0.848 & 0.831 & 0.735 \\
  uPU+TRADES ($\beta=18$)  & 0.937 & 0.870 & 0.809 & 0.735 \\
  nnPU+TRADES ($\beta=6$)  & \textbf{0.944} & \textbf{0.877} & \textbf{0.861} & 0.723 \\
  nnPU+TRADES ($\beta=12$) & 0.933 & 0.875 & 0.845 & 0.740 \\
  nnPU+TRADES ($\beta=18$) & 0.930 & 0.871 & 0.836 & \textbf{0.743} \\
 \bottomrule
 \end{tabular}
\end{table}

Table~\ref{tab:beta_sweep} shows performance as we vary the TRADES coefficient $\beta$.
On CIFAR-10, adversarial accuracy tends to increase as $\beta$ becomes larger, while clean accuracy decreases step by step.
For example, for uPU+TRADES, increasing $\beta$ from 6 to 12 improves adversarial accuracy but reduces clean accuracy, and at $\beta=18$ the adversarial accuracy appears to saturate.
Similarly for nnPU+TRADES, increasing $\beta$ improves adversarial accuracy but also induces a decrease in clean accuracy, indicating that $\beta$ is a key factor controlling the accuracy--robustness trade-off.

On F-MNIST, the variation across $\beta$ is smaller than on CIFAR-10, and in particular uPU+TRADES exhibits only limited changes in clean accuracy even as $\beta$ increases.
However, for nnPU+TRADES, setting $\beta$ too large slightly degrades adversarial accuracy, suggesting that the optimal $\beta$ may depend on the dataset and the learning scheme (uPU vs. nnPU).

% ---------------------------------------------------------
% Additional experiments for validating theoretical analysis (n_U threshold)
% ---------------------------------------------------------
\section{Validation of Theoretical Analysis}\label{subsec:validate_theory}

\paragraph{Objective.}
In Chapter~\ref{ch:theory}, as a condition under which PU learning can become advantageous compared to supervised learning (PN),
we compared the upper bounds on the estimation error of PN-TRADES and PU+TRADES,
and derived a threshold on the number of unlabeled samples, $n_{\mathrm U}^{\star}$, which holds when the upper bound for PN-TRADES exceeds that of PU+TRADES (Theorem~\ref{thm:nu-threshold-trades}).
The aim of this section is to examine to what extent this theoretical threshold aligns with experimental results as a qualitative guideline:
``increasing the amount of unlabeled data can make PU methods preferable.''
Note that the theoretical threshold is a sufficient condition based on upper-bound comparison; therefore, we do not generally expect it to exactly match the empirical turning point.

\paragraph{Validation procedure.}
We validate the theory in three steps:
(i) substitute constants into the theoretical expression to obtain a numerical value of $n_{\mathrm U}^{\star}$;
(ii) sweep $n_{\mathrm U}$ over discrete values and compare the test losses of PU+TRADES and PN-TRADES;
(iii) compare the empirical turning point $\widehat n_{\mathrm U}^{\mathrm{emp}}$ with $n_{\mathrm U}^{\star}$ and discuss reasons for the discrepancy.

% ---------------------------
% 1) Numerically instantiating n_U^*
% ---------------------------
\paragraph{Numerical instantiation of the theoretical threshold $n_{\mathrm U}^{\star}$.}
The theoretical expression contains an upper bound on the weight norm $W$ (i.e., $\|w\|_2\le W$), but the threshold used here is based on comparing estimation-error upper bounds.
We define the norm upper bound of the perturbed input as
\begin{equation}
C_x^{\mathrm{adv}} := C_x+\varepsilon d^{1/q},
\qquad
C_{\mathrm{KL}} := W C_x^{\mathrm{adv}} = W(C_x+\varepsilon d^{1/q})
\label{eq:ckl_def}
\end{equation}
and approximate an upper bound of the logistic loss by $C_\ell\approx W C_x$.
Then we can rewrite $\Gamma_\delta\approx W\,\overline{\Gamma}_\delta$, and factor out $W$ from both sides of the comparison inequality.
As a result, the numerical instantiation of the threshold simplifies to a form independent of $W$.

Below we show intermediate steps to obtain $\Gamma_\delta\approx W\,\overline{\Gamma}_\delta$.
The quantity $\Gamma_\delta$ defined in Theorem~\ref{thm:nu-threshold-trades} is
\begin{align}
\Gamma_\delta
&=
4\big(L_\ell+4\beta L_{\mathrm{KL}}\big)\,W C_x d^{1/q}
+8\beta L_{\mathrm{KL}}\,\varepsilon W d^{1/q}
+\kappa_\delta\,(C_\ell+\beta C_{\mathrm{KL}})
\label{eq:Gamma_step0_exp}
\\
&\approx
4\big(L_\ell+4\beta L_{\mathrm{KL}}\big)\,W C_x d^{1/q}
+8\beta L_{\mathrm{KL}}\,\varepsilon W d^{1/q}
+\kappa_\delta\,(W C_x+\beta W C_x^{\mathrm{adv}})
\label{eq:Gamma_step1_exp}
\\
&=
4\big(L_\ell+4\beta L_{\mathrm{KL}}\big)\,W C_x d^{1/q}
+8\beta L_{\mathrm{KL}}\,\varepsilon W d^{1/q}
+\kappa_\delta\,W\big(C_x+\beta C_x^{\mathrm{adv}}\big)
\label{eq:Gamma_step2_exp}
\\
&=
W\Big\{
4\big(L_\ell+4\beta L_{\mathrm{KL}}\big)\,C_x d^{1/q}
+8\beta L_{\mathrm{KL}}\,\varepsilon d^{1/q}
+\kappa_\delta\big(C_x+\beta C_x^{\mathrm{adv}}\big)
\Big\}
\label{eq:Gamma_step3_exp}
\\
&=: W\,\overline{\Gamma}_\delta .
\label{eq:Gamma_step4_exp}
\end{align}

Specifically, letting
\begin{equation}
\kappa_\delta:=\sqrt{2\ln\frac{2}{\delta}},
\qquad
\overline{\Gamma}_\delta
:=
4\big(L_\ell+4\beta L_{\mathrm{KL}}\big) C_x d^{1/q}
+8\beta L_{\mathrm{KL}}\varepsilon d^{1/q}
+\kappa_\delta\big(C_x+\beta C_x^{\mathrm{adv}}\big)
\label{eq:Gamma_bar_def}
\end{equation}
the feasibility condition \eqref{eq:compare-pn-uputr-feasible} in Theorem~\ref{thm:nu-threshold-trades} becomes
\begin{equation}
\overline{\Gamma}_\delta\frac{\pi_{\mathrm N}}{\sqrt{n_{\mathrm N}}}
\ >\
\frac{\pi_{\mathrm P}}{\sqrt{n_{\mathrm P}}}\Big(4L_\ell\,C_x d^{1/q}+\kappa_\delta C_x\Big),
\label{eq:compare-pn-uputr-feasible-exp}
\end{equation}
and under this condition we obtain
\begin{equation}
\boxed{\
\ n_{\mathrm U}
\ >\
\left(
\frac{\overline{\Gamma}_\delta}
{\overline{\Gamma}_\delta\frac{\pi_{\mathrm N}}{\sqrt{n_{\mathrm N}}}
-\frac{\pi_{\mathrm P}}{\sqrt{n_{\mathrm P}}}\big(4L_\ell\,C_x d^{1/q}+\kappa_\delta C_x\big)}
\right)^{\!2}
\ }
\label{eq:nu_threshold_exp_simplified}
\end{equation}
When this condition is satisfied, comparing the estimation-error upper bounds of PN-TRADES and PU+TRADES shows that the upper bound for PN-TRADES exceeds that of PU+TRADES.
That is, in theory, beyond this threshold PU+TRADES (uPU/nnPU) can become preferable to PN-TRADES.

\paragraph{Constants and settings.}
The main symbols and constants used for numerical instantiation are summarized in Table~\ref{tab:validate_theory_constants}.
In this section we use $q=2$, and for the input norm bound we set $C_x=\sqrt{d}$ (i.e., $\|x\|_2\le \sqrt{d}$).
Since $d^{1/q}=\sqrt{d}$, we have
\begin{equation}
C_x d^{1/q}=\sqrt d\cdot \sqrt d=d.
\label{eq:cx_d_simplify}
\end{equation}
Thus, we substitute $d=784$ for F-MNIST and $d=3072$ for CIFAR-10.

\begin{table}[t]
 \centering
  \caption{Constants used to numerically instantiate the theoretical threshold $n_{\mathrm U}^{\star}$ derived in Theorem~\ref{thm:nu-threshold-trades} (FashionMNIST / CIFAR-10).
We follow Table~\ref{tab:exp_settings} for the class priors and sample sizes, and summarize the input dimension $d$, norm bounds $C_x, C_x^{\mathrm{adv}}$, Lipschitz constants $L_\ell, L_{\mathrm{KL}}$, confidence level $\delta$, and so on.}
 \label{tab:validate_theory_constants}
 \vspace{4pt}
 \small
 \setlength{\tabcolsep}{4pt}
 \begin{tabular}{l l c c l}
 \toprule
  \textbf{Symbol} & \textbf{Meaning} & \textbf{FashionMNIST} & \textbf{CIFAR-10} & \textbf{Remarks} \\
 \midrule
  $\pi_{\mathrm P}$ & prior of the positive class & $0.4$ & $0.4$ & assumption \\
  $\pi_{\mathrm N}$ & prior of the negative class & $0.6$ & $0.6$ & $\pi_{\mathrm N}=1-\pi_{\mathrm P}$ \\
  $n_{\mathrm P}$ & \# positive samples & $1000$ & $1000$ & given setting \\
  $n_{\mathrm N}$ & \# negative samples & $500$ & $500$ & given setting \\
  $\beta$ & TRADES coefficient & $6$ & $6$ & representative value \\
  $\epsilon$ & perturbation radius & $0.3$ & $0.031$ & experimental setting (attack radius) \\
  $q$ & norm index & $2$ & $2$ & dual norm of $p=2$ \\
  $d$ & input dimension & $784$ & $3072$ & $1\times 28\times 28$, $3\times 32\times 32$ \\
  $C_x$ & input norm bound & $\sqrt d$ & $\sqrt d$ & $\|x\|_2\le \sqrt d$ \\
  $C_x^{\mathrm{adv}}$ & perturbed-input bound & $C_x+\epsilon d^{1/q}$ & $C_x+\epsilon d^{1/q}$ & from \eqref{eq:ckl_def} \\
  $L_\ell$ & Lipschitz constant of loss & $1$ & $1$ & logistic loss \\
  $L_{\mathrm{KL}}$ & Lipschitz constant of KL term & $1$ & $1$ & setting in Chapter~\ref{ch:theory} \\
  $\delta$ & confidence level & $0.05$ & $0.05$ & fixed \\
 \bottomrule
 \end{tabular}
\end{table}

Substituting the above settings into \eqref{eq:nu_threshold_exp_simplified}, we obtain
\[
n_{\mathrm U}^{\star}(\mathrm{F\text{-}MNIST})
\approx 1443.37
\ (\Rightarrow\ n_{\mathrm U}\ge 1444),
\]
\[
n_{\mathrm U}^{\star}(\mathrm{CIFAR\text{-}10})
\approx 1443.25
\ (\Rightarrow\ n_{\mathrm U}\ge 1444).
\]

% ---------------------------
% 2) Experimental protocol
% ---------------------------
\paragraph{Experimental protocol for validation.}
Since the theoretical comparison concerns estimation error (risk), in this section we use the loss on the test set as a proxy.
Specifically, for each $n_{\mathrm U}$, let
\(
t^{\star}(n_{\mathrm U})
\)
denote the epoch achieving the highest Clean Accuracy, and we compare the sum of the clean loss and adversarial loss at that epoch:
\begin{equation}
\mathcal{L}_{\mathrm{sum}}(n_{\mathrm U})
:=
\mathcal{L}_{\mathrm{clean}}\big(t^{\star}(n_{\mathrm U});n_{\mathrm U}\big)
+
\mathcal{L}_{\mathrm{adv}}\big(t^{\star}(n_{\mathrm U});n_{\mathrm U}\big).
\label{eq:loss_sum_def}
\end{equation}
Here, $\mathcal{L}_{\mathrm{clean}}$ is the clean loss on the test set, and $\mathcal{L}_{\mathrm{adv}}$ is the adversarial loss computed on adversarial examples generated from the same test set.
We consider the following settings: (a) FMNIST: 6-layer MLP and a linear model; (b) CIFAR-10: a linear model and ResNet-18.
For each setting, we train PU+TRADES and PN-TRADES and compare how $\mathcal{L}_{\mathrm{sum}}$ changes with $n_{\mathrm U}$.

\paragraph{Empirical turning point.}
To relate to the theoretical threshold $n_{\mathrm U}^{\star}$, we define the empirical turning point as the smallest $n_{\mathrm U}$ at which the sign of
\[
\Delta(n_{\mathrm U})
:=
\mathcal{L}_{\mathrm{sum}}^{\mathrm{PU\text{-}TRADES}}(n_{\mathrm U})
-
\mathcal{L}_{\mathrm{sum}}^{\mathrm{PN\text{-}TRADES}}(n_{\mathrm U})
\]
becomes negative.
Moreover, if the sign of $\Delta$ flips between $n_{\mathrm U}=U_k$ and $U_{k+1}$, we estimate the turning point by linear interpolation:
\begin{equation}
\widehat n_{\mathrm U}^{\mathrm{emp}}
:=
U_k
+
\frac{-\Delta(U_k)}{\Delta(U_{k+1})-\Delta(U_k)}(U_{k+1}-U_k).
\label{eq:nu_emp_def}
\end{equation}

% ---------------------------
% 3) Results
% ---------------------------
\paragraph{Results.}
Figure~\ref{fig:validate_theory_loss_sum} shows the trajectories of $\mathcal{L}_{\mathrm{sum}}$ as a function of $n_{\mathrm U}$.
The orange curve represents $\mathcal{L}_{\mathrm{sum}}(n_{\mathrm U})$ for PU+TRADES, the dashed gray line is the baseline value for PN-TRADES,
and the blue vertical line indicates the theoretical threshold $n_{\mathrm U}^{\star}$.
When a crossing is observed within the sweep range, we mark $\widehat n_{\mathrm U}^{\mathrm{emp}}$ estimated by \eqref{eq:nu_emp_def} as a point in the figure.

\begin{figure}[t]
 \centering
 \begin{minipage}{0.48\linewidth}
 \centering
 \includegraphics[width=\linewidth]{fmnist_linear2.undefined.pdf}
 \caption*{(a) FMNIST / Linear}
 \end{minipage}\hfill
 \begin{minipage}{0.48\linewidth}
 \centering
 \includegraphics[width=\linewidth]{fmnist_6lMLP2.undefined.pdf}
 \caption*{(b) FMNIST / 6-layer MLP}
 \end{minipage}

 \vspace{0.5em}

 \begin{minipage}{0.48\linewidth}
 \centering
 \includegraphics[width=\linewidth]{cifar10_linear.undefined.pdf}
 \caption*{(c) CIFAR-10 / Linear}
 \end{minipage}\hfill
 \begin{minipage}{0.48\linewidth}
 \centering
 \includegraphics[width=\linewidth]{cifar10_resnet.undefined.pdf}
 \caption*{(d) CIFAR-10 / ResNet-18}
 \end{minipage}
  \caption{Validation results for the theoretical threshold $n_{\mathrm U}^{\star}$ of the unlabeled sample size derived in our theoretical analysis.
Each panel plots the test loss sum for PU+TRADES,
$\mathcal{L}_{\mathrm{sum}}(n_{\mathrm U})=\mathcal{L}_{\mathrm{clean}}(t^{\star}(n_{\mathrm U});n_{\mathrm U})+\mathcal{L}_{\mathrm{adv}}(t^{\star}(n_{\mathrm U});n_{\mathrm U})$ (Eq.~\eqref{eq:loss_sum_def}),
against $n_{\mathrm U}$, and compares it with the PN-TRADES baseline value (which does not depend on $n_{\mathrm U}$).
Here, $t^{\star}(n_{\mathrm U})$ denotes the epoch achieving the highest Clean Accuracy.
The intersection $\widehat n_{\mathrm U}^{\mathrm{emp}}$ is estimated by linear interpolation.}
 \label{fig:validate_theory_loss_sum}
\end{figure}

\begin{table}[t]
 \centering
  \caption{Comparison between the theoretical threshold $n_{\mathrm U}^{\star}$ (Theorem~\ref{thm:nu-threshold-trades}) and the empirical turning point $\widehat n_{\mathrm U}^{\mathrm{emp}}$.
$\widehat n_{\mathrm U}^{\mathrm{emp}}$ is estimated by linear interpolation from Fig.~\ref{fig:validate_theory_loss_sum} as the value of $n_{\mathrm U}$ where the loss sums $\mathcal{L}_{\mathrm{sum}}$ of PU+TRADES and PN-TRADES coincide.
``---'' indicates that no intersection existed within the range of $n_{\mathrm U}$ evaluated in this study.}
 \label{tab:validate_theory_summary}
 \begin{tabular}{lcc}
 \toprule
  Setting & $n_{\mathrm U}^{\star}$ (theory) & $\widehat n_{\mathrm U}^{\mathrm{emp}}$ (emp.) \\
 \midrule
  FMNIST / 6-layer MLP & 1444 & 4788.14 \\
  FMNIST / Linear  & 1444 & 4247.13 \\
  CIFAR-10 / Linear  & 1444 & 2750.00 \\
  CIFAR-10 / ResNet-18  & 1444 & --- \\
 \bottomrule
 \end{tabular}
\end{table}

\paragraph{Discussion.}
From Fig.~\ref{fig:validate_theory_loss_sum} and Table~\ref{tab:validate_theory_summary}, for F-MNIST (both the linear model and the 6-layer MLP),
$\mathcal{L}_{\mathrm{sum}}(n_{\mathrm U})$ decreases as $n_{\mathrm U}$ increases, and the intersection where the loss sums of PU+TRADES and PN-TRADES coincide is estimated as
$\widehat n_{\mathrm U}^{\mathrm{emp}}\approx 4247.13$ (linear model) and $\widehat n_{\mathrm U}^{\mathrm{emp}}\approx 4788.14$ (6-layer MLP).
These values are larger than the theoretical threshold $n_{\mathrm U}^{\star}=1444$,
which is consistent with the fact that $n_{\mathrm U}^{\star}$ is a sufficient condition derived from upper-bound comparison (and hence can be conservative).

For CIFAR-10, the results depend on the model:
(i) \textbf{CIFAR-10 / Linear}:
a crossing is observed at $\widehat n_{\mathrm U}^{\mathrm{emp}}\approx 2750.00$ (between $U=2000$ and $3000$).
(ii) \textbf{CIFAR-10 / ResNet-18}:
using the PN baseline value $1.1604$, the PU loss sum remains smaller throughout the range $U=1000$--$10000$,
and thus there is no intersection within this range (i.e., PU is already better at the smallest observed value $U=1000$).
Overall, while the theoretical threshold $n_{\mathrm U}^{\star}$ is not an exact predictor of the intersection location,
the experiments also indicate that with sufficiently many unlabeled samples, PU methods can outperform PN methods.

\section{Conclusion}\label{sec:conclusion}

In this study, we focus on Positive–Unlabeled (PU) learning, which trains a classifier using only positive and unlabeled data.
Motivated by applications such as medical image analysis—where fully annotating data is difficult and even small perturbations can lead to critical misclassifications—we aim to improve robustness against adversarial perturbations.
First, we confirm that naively applying adversarial training to PU learning can make optimization unstable and significantly degrade clean performance, because unlabeled data, despite containing a mixture of positive and negative samples, tends to be treated uniformly as negative in the loss.
To address this issue, we propose \textbf{PU+TRADES}, an extension of TRADES, a representative adversarial training framework, by integrating a PU loss with a label-independent regularization term.
Experiments on multiple benchmark datasets and medical imaging data demonstrate that while standard PU learning can achieve high clean accuracy, its adversarial accuracy may drop drastically; in contrast, the proposed method improves adversarial accuracy without substantially sacrificing clean accuracy.
Moreover, we show that adjusting the TRADES coefficient allows one to control the trade-off between robustness and clean performance.
In addition, we derive an upper bound on the estimation error for binary classification under adversarial perturbations, and by comparing supervised learning with PU learning, we theoretically clarify conditions under which PU learning is advantageous, expressed as an inequality involving the numbers of positive (P), negative (N), and unlabeled (U) samples.
Future work includes designing robust learning schemes that incorporate estimation error when the class prior is unknown, and extending the theory to deep models beyond linear assumptions.

\begin{appendices}

    \section{Proofs of Theoretical Analysis}
    \label{app:theory-proofs}
    
    \subsection{Background / Tool Box (Standard Results)}
    \label{app:toolbox}
    
    In this appendix, we summarize standard lemmas and inequalities used in the proofs of this chapter.
    
    \begin{tcolorbox}[
        colback = white,
        colframe = black!60,
        boxrule = 0.35pt,
        fonttitle = \bfseries,
        breakable = false]
    \begin{lem}[Talagrand's Contraction Lemma]\label{lem:talagrand_contraction}
    Let $S_n=\{\bm{x}_i\}_{i=1}^n \stackrel{\text{i.i.d.}}{\sim} \nu(\bm{x})$.
    Suppose $f:\mathbb{R}\to\mathbb{R}$ is $L_f$-Lipschitz.
    Then, for the Rademacher complexities of $\mathscr{G}$ and $f\circ\mathscr{G}$,
    \begin{equation}
    \mathfrak{R}_{n, \nu}(f \circ \mathscr{G}) \leq L_f \, \mathfrak{R}_{n, \nu}(\mathscr{G})
    \label{eq:radcontraction}
    \end{equation}
    holds.
    \end{lem}
    \end{tcolorbox}
    
    \begin{tcolorbox}[
        colback = white,
        colframe = black!60,
        boxrule = 0.35pt,
        fonttitle = \bfseries,
        breakable = false]
    \begin{lem}[Vector Contraction]\label{lem:vecContract}
    If $\ell(u,v)$ is $L_\ell$-Lipschitz in each argument, then for any pair of functions $(f_h,g_h)$,
    \begin{equation}
    \begin{aligned}
    \mathbb{E}_{\sigma}\Big[\sup_{h}\frac{1}{n}\sum_{i=1}^n \sigma_i\,
    \ell\big(f_h(\bm x_i),\,g_h(\bm x_i)\big)\Big]
    &\le 2L_\ell\Bigg\{
    \mathbb{E}_{\sigma}\Big[\sup_{h}\frac{1}{n}\sum_{i=1}^n \sigma_i f_h(\bm x_i)\Big]
    \\
    &\quad
    + \mathbb{E}_{\sigma}\Big[\sup_{h}\frac{1}{n}\sum_{i=1}^n \sigma_i g_h(\bm x_i)\Big]
    \Bigg\}.
    \end{aligned}
    \end{equation}
    \end{lem}
    \end{tcolorbox}
    
    \begin{tcolorbox}[
        colback = white,
        colframe = black!60,
        boxrule = 0.35pt,
        fonttitle = \bfseries,
        breakable = false]
    \begin{thm}[Upper Bound on Rademacher Complexity]\label{thm:rademacher_bound}
    Assume the input satisfies $\|\bm{x}\|_\infty \le C_x$, and let the class of linear classifiers be
    $\mathscr{G}=\{\bm{x}\mapsto \bm{w}^{\top}\bm{x}\mid \|\bm{w}\|_\infty \le W\}$.
    Then,
    \begin{equation}
    \mathfrak{R}_{n, \nu}\!\left(\mathscr{G}\right) \leq \frac{C_x W}{\sqrt{n}}
    \end{equation}
    holds.
    \end{thm}
    \end{tcolorbox}
    
    \begin{tcolorbox}[
        colback = white,
        colframe = black!60,
        boxrule = 0.35pt,
        fonttitle = \bfseries,
        breakable = false]
    \begin{lem}[Adversarial Rademacher Additive Term]\label{lem:advRad}
    For the linear class
    $\mathscr{G}=\{\bm x\mapsto \bm w^\top\bm x:\ \|\bm w\|_p\le W\}$,
    for any distribution $\nu$ and any $n\in\mathbb{N}$,
    \[
    \mathfrak{R}_{n,\nu}\Big(\{\,\bm x\mapsto g(\bm x+\bm \eta):\ \|\bm \eta\|_\infty\le \varepsilon,\ g\in\mathscr{G}\,\}\Big)
    \ \le\ \mathfrak{R}_{n,\nu}(\mathscr{G})\ +\ \frac{\varepsilon W\, d^{1/q}}{\sqrt{n}}.
    \]
    \end{lem}
    \end{tcolorbox}
    
    \begin{tcolorbox}[
        colback = white,
        colframe = black!60,
        boxrule = 0.35pt,
        fonttitle = \bfseries,
        breakable = false]
    
        \begin{thm}[McDiarmid's Inequality]\label{thm:mcdiarmid}
            Let $X_1,\ldots,X_n$ be independent random variables taking values in a set $\mathcal{X}$, and consider a function $f:\mathcal{X}^n \to \mathbb{R}$.
            Assume that there exist constants $c_1,\ldots,c_n$ such that, for each $i=1,\ldots,n$ and any
            $\bm{x}_1,\ldots,\bm{x}_n,\bm{x}_i'\in\mathcal{X}$,
            \begin{equation}
            \left|f\left(\bm{x}_1,\ldots,\bm{x}_i,\ldots,\bm{x}_n\right)
            -f\left(\bm{x}_1,\ldots,\bm{x}_i',\ldots,\bm{x}_n\right)\right|
            \le c_i
            \label{eq:data_replacement}
            \end{equation}
            holds.
            Then, for any $t>0$,
            \begin{equation}
            \Pr\!\left(
            f\left(X_1,\ldots,X_n\right)-\mathbb{E}\left[f\left(X_1,\ldots,X_n\right)\right]\ge t
            \right)
            \le
            \exp\!\left(-\frac{2t^2}{\sum_{i=1}^n c_i^2}\right)
            \label{eq:mcdiarmid}
            \end{equation}
            holds.
            \end{thm}
            \end{tcolorbox}
            
            
            \subsection{Proof of the Lemma ``Uniform Deviation Bound for Supervised TRADES'' (Lemma~\ref{lem:pntr-unif-jp})}
            \label{app:proof-lem-pntr-unif-jp}
            \begin{proof}
            We divide the proof into the following steps (i)--(iv).
            We evaluate
            $\sup_{g\in\mathscr G}\{\widehat R_{\mathrm{PN\mbox{-}TR}}(g)-R_{\mathrm{PN\mbox{-}TR}}(g)\}$ and
            $\sup_{g\in\mathscr G}\{R_{\mathrm{PN\mbox{-}TR}}(g)-\widehat R_{\mathrm{PN\mbox{-}TR}}(g)\}$
            with probability at least $1-\delta/2$, respectively, and then combine them via a union bound to obtain an upper bound on the absolute value.
            Below, we only show
            $\sup_{g\in\mathscr G}\{\widehat R_{\mathrm{PN\mbox{-}TR}}(g)-R_{\mathrm{PN\mbox{-}TR}}(g)\}$
            (the reverse direction can be proved analogously).
            
            \paragraph{\textbf{(i) McDiarmid's inequality}}
            Since $\ell(\cdot)\le C_\ell$ and $\ell_{\mathrm{KL}}(\cdot)\le C_{\mathrm{KL}}$,
            replacing one sample on the P side changes $\widehat R_{\mathrm{PN\mbox{-}TR}}(g)$ by at most
            $\frac{\pi_{\mathrm P}}{n_{\mathrm P}}(C_\ell+\beta C_{\mathrm{KL}})$,
            and replacing one sample on the N side changes it by at most
            $\frac{\pi_{\mathrm N}}{n_{\mathrm N}}(C_\ell+\beta C_{\mathrm{KL}})$.
            Therefore, by McDiarmid's inequality~\eqref{thm:mcdiarmid}, for any $t>0$,
            \begin{equation}
            \begin{aligned}
            \Pr\!\Big(
            \sup_{g\in\mathscr G}\{\widehat R_{\mathrm{PN\mbox{-}TR}}(g)-R_{\mathrm{PN\mbox{-}TR}}(g)\}
            -\mathbb{E}\big[\sup_{g\in\mathscr G}\{\widehat R_{\mathrm{PN\mbox{-}TR}}(g)-R_{\mathrm{PN\mbox{-}TR}}(g)\}\big]
            \ge t
            \Big)
            \\
            \le
            \exp\!\Bigg(
            - \frac{2t^2}
            {(C_\ell+\beta C_{\mathrm{KL}})^2
            (\pi_{\mathrm P}^2/n_{\mathrm P}+\pi_{\mathrm N}^2/n_{\mathrm N})}
            \Bigg).
            \end{aligned}
            \end{equation}
            Setting
            $\delta/2=\exp\left(
            - 2 t^2
            / \bigl((C_\ell+\beta C_{\mathrm{KL}})^2(\pi_{\mathrm P}^2/n_{\mathrm P}+\pi_{\mathrm N}^2/n_{\mathrm N})\bigr)
            \right)$
            and solving for $t$, and then using the subadditivity of the square root, we obtain, with probability at least $1-\delta/2$,
            \begin{equation}
            \begin{aligned}
            \sup_{g\in\mathscr G}\{\widehat R_{\mathrm{PN\mbox{-}TR}}(g)-R_{\mathrm{PN\mbox{-}TR}}(g)\}
            & \le
            \mathbb{E}\big[\sup_{g\in\mathscr G}\{\widehat R_{\mathrm{PN\mbox{-}TR}}(g)-R_{\mathrm{PN\mbox{-}TR}}(g)\}\big] \\
            & +
            \sqrt{\frac{1}{2}\ln\frac{2}{\delta}}\,(C_\ell+\beta C_{\mathrm{KL}})
            \left(\frac{\pi_{\mathrm P}}{\sqrt{n_{\mathrm P}}}+\frac{\pi_{\mathrm N}}{\sqrt{n_{\mathrm N}}}\right)
            \label{eq:pntr-mcdiarmid-root-jp}
            \end{aligned}
            \end{equation}
            as desired.
            
            \paragraph{\textbf{(ii) Ghost sampling and symmetrization}}
We use the symmetrization technique in statistical learning theory~\cite{Vapnik1998}.
First, in
\[
\mathbb{E}\Big[\sup_{g\in\mathscr G}\ \widehat R_{\mathrm{PN\mbox{-}TR}}(g)-R_{\mathrm{PN\mbox{-}TR}}(g)\Big],
\]
the expectation $\mathbb{E}$ is taken over repeated sampling of $(\mathscr X_{\mathrm P},\mathscr X_{\mathrm N})$ used to evaluate $\widehat R_{\mathrm{PN\mbox{-}TR}}(g)$.
To make this explicit, we write
\[
\widehat R_{\mathrm{PN\mbox{-}TR}}(g)=\widehat R_{\mathrm{PN\mbox{-}TR}}(g;\mathscr X_{\mathrm P},\mathscr X_{\mathrm N}).
\]
Let $(\mathscr X_{\mathrm P}',\mathscr X_{\mathrm N}')$ be a ghost sample (an independent copy with the same distribution) independent of $(\mathscr X_{\mathrm P},\mathscr X_{\mathrm N})$.
Since the true risk can be written as the expectation over the ghost sample,
\[
R_{\mathrm{PN\mbox{-}TR}}(g)
=\mathbb{E}_{(\mathscr X_{\mathrm P}',\mathscr X_{\mathrm N}')}\big[
\widehat R_{\mathrm{PN\mbox{-}TR}}(g;\mathscr X_{\mathrm P}',\mathscr X_{\mathrm N}')
\big],
\]
we obtain
\begin{align}
&\mathbb{E}_{(\mathscr X_{\mathrm P},\mathscr X_{\mathrm N})}
\Big[\sup_{g\in\mathscr G}\ \widehat R_{\mathrm{PN\mbox{-}TR}}(g;\mathscr X_{\mathrm P},\mathscr X_{\mathrm N})
- R_{\mathrm{PN\mbox{-}TR}}(g)\Big]
\nonumber\\
&=
\mathbb{E}_{(\mathscr X_{\mathrm P},\mathscr X_{\mathrm N})}
\Big[\sup_{g\in\mathscr G}\ \widehat R_{\mathrm{PN\mbox{-}TR}}(g;\mathscr X_{\mathrm P},\mathscr X_{\mathrm N})
-\mathbb{E}_{(\mathscr X_{\mathrm P}',\mathscr X_{\mathrm N}')}
\widehat R_{\mathrm{PN\mbox{-}TR}}(g;\mathscr X_{\mathrm P}',\mathscr X_{\mathrm N}')\Big]
\nonumber\\
&\le
\mathbb{E}_{(\mathscr X_{\mathrm P},\mathscr X_{\mathrm N}),(\mathscr X_{\mathrm P}',\mathscr X_{\mathrm N}')}
\Big[\sup_{g\in\mathscr G}\
\widehat R_{\mathrm{PN\mbox{-}TR}}(g;\mathscr X_{\mathrm P},\mathscr X_{\mathrm N})
-
\widehat R_{\mathrm{PN\mbox{-}TR}}(g;\mathscr X_{\mathrm P}',\mathscr X_{\mathrm N}')
\Big],
\label{eq:pntr-ghost-sym}
\end{align}
where we used Jensen's inequality, since $\sup$ is a convex function.

Next, we decompose the difference into the P and N parts.
For notational simplicity, define
\[
\phi_y(g,\bm x)
:=
\ell\big(g(\bm x),y\big)
+
\beta\max_{\|\bm\eta\|_\infty\le\varepsilon}
\ell_{\mathrm{KL}}\big(g(\bm x),g(\bm x+\bm\eta)\big).
\]
Then,
\begin{align}
&\widehat R_{\mathrm{PN\mbox{-}TR}}(g;\mathscr X_{\mathrm P},\mathscr X_{\mathrm N})
-
\widehat R_{\mathrm{PN\mbox{-}TR}}(g;\mathscr X_{\mathrm P}',\mathscr X_{\mathrm N}')
\nonumber\\
&=
\frac{\pi_{\mathrm P}}{n_{\mathrm P}}\sum_{i=1}^{n_{\mathrm P}}
\Big(\phi_{+1}(g,\bm x_i^{\mathrm P})-\phi_{+1}(g,\bm x_i^{\mathrm P\prime})\Big)
+
\frac{\pi_{\mathrm N}}{n_{\mathrm N}}\sum_{i=1}^{n_{\mathrm N}}
\Big(\phi_{-1}(g,\bm x_i^{\mathrm N})-\phi_{-1}(g,\bm x_i^{\mathrm N\prime})\Big).
\label{eq:pntr-diff-decomp}
\end{align}
By the subadditivity of $\sup$,
\begin{align}
&\mathbb{E}_{(\mathscr X_{\mathrm P},\mathscr X_{\mathrm N}),(\mathscr X_{\mathrm P}',\mathscr X_{\mathrm N}')}
\Big[\sup_{g\in\mathscr G}\
\widehat R_{\mathrm{PN\mbox{-}TR}}(g;\mathscr X_{\mathrm P},\mathscr X_{\mathrm N})
-
\widehat R_{\mathrm{PN\mbox{-}TR}}(g;\mathscr X_{\mathrm P}',\mathscr X_{\mathrm N}')
\Big]
\nonumber\\
&\le
\frac{\pi_{\mathrm P}}{n_{\mathrm P}}
\mathbb{E}_{\mathscr X_{\mathrm P},\mathscr X_{\mathrm P}'}
\Big[\sup_{g\in\mathscr G}\sum_{i=1}^{n_{\mathrm P}}
\big(\phi_{+1}(g,\bm x_i^{\mathrm P})-\phi_{+1}(g,\bm x_i^{\mathrm P\prime})\big)\Big]
\nonumber\\
&\quad+
\frac{\pi_{\mathrm N}}{n_{\mathrm N}}
\mathbb{E}_{\mathscr X_{\mathrm N},\mathscr X_{\mathrm N}'}
\Big[\sup_{g\in\mathscr G}\sum_{i=1}^{n_{\mathrm N}}
\big(\phi_{-1}(g,\bm x_i^{\mathrm N})-\phi_{-1}(g,\bm x_i^{\mathrm N\prime})\big)\Big].
\label{eq:pntr-sup-subadd}
\end{align}

Now consider, for example, the P side.
Since $\bm x_i^{\mathrm P}$ and $\bm x_i^{\mathrm P\prime}$ are independent and identically distributed (both from $p_{\mathrm P}$),
the two differences
$\phi_{+1}(g,\bm x_i^{\mathrm P})-\phi_{+1}(g,\bm x_i^{\mathrm P\prime})$
and
$\phi_{+1}(g,\bm x_i^{\mathrm P\prime})-\phi_{+1}(g,\bm x_i^{\mathrm P})$
have the same distribution.
Therefore, letting
$L_{2,n_{\mathrm P}}:=\sum_{i=2}^{n_{\mathrm P}}\big(\phi_{+1}(g,\bm x_i^{\mathrm P})-\phi_{+1}(g,\bm x_i^{\mathrm P\prime})\big)$, we have
\begin{align}
&\mathbb{E}_{\mathscr X_{\mathrm P},\mathscr X_{\mathrm P}'}
\Big[\sup_{g\in\mathscr G}\sum_{i=1}^{n_{\mathrm P}}
\big(\phi_{+1}(g,\bm x_i^{\mathrm P})-\phi_{+1}(g,\bm x_i^{\mathrm P\prime})\big)\Big]
\nonumber\\
&=
\mathbb{E}_{\mathscr X_{\mathrm P},\mathscr X_{\mathrm P}'}
\Big[\sup_{g\in\mathscr G}
\big(\phi_{+1}(g,\bm x_1^{\mathrm P})-\phi_{+1}(g,\bm x_1^{\mathrm P\prime})\big)+L_{2,n_{\mathrm P}}\Big]
\nonumber\\
&=
\frac12\mathbb{E}_{\mathscr X_{\mathrm P},\mathscr X_{\mathrm P}'}
\Big[\sup_{g\in\mathscr G}
\big(\phi_{+1}(g,\bm x_1^{\mathrm P})-\phi_{+1}(g,\bm x_1^{\mathrm P\prime})\big)+L_{2,n_{\mathrm P}}\Big]
\nonumber\\
&\quad+
\frac12\mathbb{E}_{\mathscr X_{\mathrm P},\mathscr X_{\mathrm P}'}
\Big[\sup_{g\in\mathscr G}
\big(\phi_{+1}(g,\bm x_1^{\mathrm P\prime})-\phi_{+1}(g,\bm x_1^{\mathrm P})\big)+L_{2,n_{\mathrm P}}\Big]
\nonumber\\
&=
\mathbb{E}_{\sigma_1,\mathscr X_{\mathrm P},\mathscr X_{\mathrm P}'}
\Big[\sup_{g\in\mathscr G}
\sigma_1\big(\phi_{+1}(g,\bm x_1^{\mathrm P})-\phi_{+1}(g,\bm x_1^{\mathrm P\prime})\big)+L_{2,n_{\mathrm P}}\Big].
\label{eq:pntr-sym-stepP}
\end{align}
Repeating the same argument $n_{\mathrm P}$ times yields
\begin{align}
&\mathbb{E}_{\mathscr X_{\mathrm P},\mathscr X_{\mathrm P}'}
\Big[\sup_{g\in\mathscr G}\sum_{i=1}^{n_{\mathrm P}}
\big(\phi_{+1}(g,\bm x_i^{\mathrm P})-\phi_{+1}(g,\bm x_i^{\mathrm P\prime})\big)\Big]
\nonumber\\
&=
\mathbb{E}_{\bm\sigma,\mathscr X_{\mathrm P},\mathscr X_{\mathrm P}'}
\Big[\sup_{g\in\mathscr G}\sum_{i=1}^{n_{\mathrm P}}
\sigma_i\big(\phi_{+1}(g,\bm x_i^{\mathrm P})-\phi_{+1}(g,\bm x_i^{\mathrm P\prime})\big)\Big]
\nonumber\\
&\le
\mathbb{E}_{\bm\sigma,\mathscr X_{\mathrm P}}
\Big[\sup_{g\in\mathscr G}\sum_{i=1}^{n_{\mathrm P}}
\sigma_i\,\phi_{+1}(g,\bm x_i^{\mathrm P})\Big]
+
\mathbb{E}_{\bm\sigma,\mathscr X_{\mathrm P}'}
\Big[\sup_{g\in\mathscr G}\sum_{i=1}^{n_{\mathrm P}}
(-\sigma_i)\,\phi_{+1}(g,\bm x_i^{\mathrm P\prime})\Big]
\nonumber\\
&=
2\,\mathbb{E}_{\bm\sigma,\mathscr X_{\mathrm P}}
\Big[\sup_{g\in\mathscr G}\sum_{i=1}^{n_{\mathrm P}}
\sigma_i\,\phi_{+1}(g,\bm x_i^{\mathrm P})\Big]
=
2n_{\mathrm P}\,\mathfrak{R}_{n_{\mathrm P},p_{\mathrm P}}(\phi_{+1}\circ\mathscr G).
\label{eq:pntr-rad-phiP}
\end{align}
Similarly, for the N side,
\[
\mathbb{E}_{\mathscr X_{\mathrm N},\mathscr X_{\mathrm N}'}
\Big[\sup_{g\in\mathscr G}\sum_{i=1}^{n_{\mathrm N}}
\big(\phi_{-1}(g,\bm x_i^{\mathrm N})-\phi_{-1}(g,\bm x_i^{\mathrm N\prime})\big)\Big]
\le
2n_{\mathrm N}\,\mathfrak{R}_{n_{\mathrm N},p_{\mathrm N}}(\phi_{-1}\circ\mathscr G)
\]
is obtained.

Finally, we upper bound $\mathfrak{R}(\phi_y\circ\mathscr G)$ by $\mathfrak{R}(\mathscr G)$.
Write $\phi_y=\ell(\cdot,y)+\beta\,\psi$, where
\[
\psi(g,\bm x):=\max_{\|\bm\eta\|_\infty\le\varepsilon}
\ell_{\mathrm{KL}}\big(g(\bm x),g(\bm x+\bm\eta)\big).
\]
Then, by the subadditivity of $\sup$,
\[
\mathfrak{R}_{n,q}(\phi_y\circ\mathscr G)
\le
\mathfrak{R}_{n,q}(\ell(\cdot,y)\circ\mathscr G)
+\beta\,\mathfrak{R}_{n,q}(\psi\circ\mathscr G).
\]
For the classification loss term, by the contraction lemma,
\[
\mathfrak{R}_{n,q}(\ell(\cdot,y)\circ\mathscr G)
\le
L_\ell\,\mathfrak{R}_{n,q}(\mathscr G).
\]
For the KL term, applying Lemma~\ref{lem:vecContract} to
$f_g(\bm x)=g(\bm x)$ and $g_{g,\bm\eta}(\bm x)=g(\bm x+\bm\eta)$,
and then using Lemma~\ref{lem:advRad}, we obtain
\begin{align}
\mathfrak{R}_{n,q}(\psi\circ\mathscr G)
&\le
2L_{\mathrm{KL}}
\Big(
\mathfrak{R}_{n,q}(\mathscr G)
+
\mathfrak{R}_{n,q}(\{\,\bm x\mapsto g(\bm x+\bm\eta):\ \|\bm\eta\|_\infty\le\varepsilon,\ g\in\mathscr G\,\})
\Big)
\nonumber\\
&\le
2L_{\mathrm{KL}}
\Big(
2\mathfrak{R}_{n,q}(\mathscr G)
+\frac{\varepsilon W d^{1/q}}{\sqrt n}
\Big).
\label{eq:pntr-rad-kl-bound}
\end{align}
Substituting these bounds into \eqref{eq:pntr-ghost-sym}--\eqref{eq:pntr-sup-subadd}, we get
\begin{align}
&\mathbb{E}\Big[\sup_{g\in\mathscr G}\ \widehat R_{\mathrm{PN\mbox{-}TR}}(g)-R_{\mathrm{PN\mbox{-}TR}}(g)\Big]
\nonumber\\
&\le
\frac{\pi_{\mathrm P}}{n_{\mathrm P}}
\cdot 2n_{\mathrm P}\,\mathfrak{R}_{n_{\mathrm P},p_{\mathrm P}}(\phi_{+1}\circ\mathscr G)
\;+\;
\frac{\pi_{\mathrm N}}{n_{\mathrm N}}
\cdot 2n_{\mathrm N}\,\mathfrak{R}_{n_{\mathrm N},p_{\mathrm N}}(\phi_{-1}\circ\mathscr G)
\nonumber\\
&=
2\pi_{\mathrm P}\,\mathfrak{R}_{n_{\mathrm P},p_{\mathrm P}}(\phi_{+1}\circ\mathscr G)
+
2\pi_{\mathrm N}\,\mathfrak{R}_{n_{\mathrm N},p_{\mathrm N}}(\phi_{-1}\circ\mathscr G)
\nonumber\\
&\le
2\pi_{\mathrm P}\Big\{
L_\ell\,\mathfrak{R}_{n_{\mathrm P},p_{\mathrm P}}(\mathscr G)
+\beta\,\mathfrak{R}_{n_{\mathrm P},p_{\mathrm P}}(\psi\circ\mathscr G)
\Big\}
+
2\pi_{\mathrm N}\Big\{
L_\ell\,\mathfrak{R}_{n_{\mathrm N},p_{\mathrm N}}(\mathscr G)
+\beta\,\mathfrak{R}_{n_{\mathrm N},p_{\mathrm N}}(\psi\circ\mathscr G)
\Big\}
\nonumber\\
&\le
2\pi_{\mathrm P}\Big\{
L_\ell\,\mathfrak{R}_{n_{\mathrm P},p_{\mathrm P}}(\mathscr G)
+\beta\cdot 2L_{\mathrm{KL}}
\Big(
2\mathfrak{R}_{n_{\mathrm P},p_{\mathrm P}}(\mathscr G)
+\frac{\varepsilon W d^{1/q}}{\sqrt{n_{\mathrm P}}}
\Big)
\Big\}
\nonumber\\
&\quad+
2\pi_{\mathrm N}\Big\{
L_\ell\,\mathfrak{R}_{n_{\mathrm N},p_{\mathrm N}}(\mathscr G)
+\beta\cdot 2L_{\mathrm{KL}}
\Big(
2\mathfrak{R}_{n_{\mathrm N},p_{\mathrm N}}(\mathscr G)
+\frac{\varepsilon W d^{1/q}}{\sqrt{n_{\mathrm N}}}
\Big)
\Big\}
\nonumber\\
&=
2\big(L_\ell+4\beta L_{\mathrm{KL}}\big)
\Big(
\pi_{\mathrm P}\mathfrak{R}_{n_{\mathrm P},p_{\mathrm P}}(\mathscr G)
+
\pi_{\mathrm N}\mathfrak{R}_{n_{\mathrm N},p_{\mathrm N}}(\mathscr G)
\Big)
+
4\beta L_{\mathrm{KL}}\,\varepsilon W d^{1/q}
\Big(
\frac{\pi_{\mathrm P}}{\sqrt{n_{\mathrm P}}}
+
\frac{\pi_{\mathrm N}}{\sqrt{n_{\mathrm N}}}
\Big),
\label{eq:pntr-exp-sup-bound}
\end{align}
where
$\phi_y(g,\bm x):=\ell(g(\bm x),y)+\beta\max_{\|\bm\eta\|_\infty\le\varepsilon}
\ell_{\mathrm{KL}}(g(\bm x),g(\bm x+\bm\eta))$
and
$\psi(g,\bm x):=\max_{\|\bm\eta\|_\infty\le\varepsilon}
\ell_{\mathrm{KL}}(g(\bm x),g(\bm x+\bm\eta))$.

\paragraph{\textbf{(iii) Combining the results of (i) and (ii)}}
Substituting the expectation bound in (ii), i.e., \eqref{eq:pntr-exp-sup-bound}, into McDiarmid's inequality result in (i), i.e., \eqref{eq:pntr-mcdiarmid-root-jp}, we obtain, with probability at least $1-\delta/2$,
\begin{align}
\sup_{g\in\mathscr G}\Big\{\widehat R_{\mathrm{PN\mbox{-}TR}}(g)-R_{\mathrm{PN\mbox{-}TR}}(g)\Big\}
\ \le\
&
2\bigl(L_\ell+4\beta L_{\mathrm{KL}}\bigr)
\Big(
\pi_{\mathrm P}\mathfrak{R}_{n_{\mathrm P},p_{\mathrm P}}(\mathscr G)
+
\pi_{\mathrm N}\mathfrak{R}_{n_{\mathrm N},p_{\mathrm N}}(\mathscr G)
\Big)
\nonumber\\
&+
4\beta L_{\mathrm{KL}}\,\varepsilon W d^{1/q}\left(\frac{\pi_{\mathrm P}}{\sqrt{n_{\mathrm P}}}+\frac{\pi_{\mathrm N}}{\sqrt{n_{\mathrm N}}}\right)
\nonumber\\
&+
\sqrt{\frac{1}{2}\ln\frac{2}{\delta}}\,(C_\ell+\beta C_{\mathrm{KL}})
\left(\frac{\pi_{\mathrm P}}{\sqrt{n_{\mathrm P}}}+\frac{\pi_{\mathrm N}}{\sqrt{n_{\mathrm N}}}\right)
\label{eq:pntr-sup-one-dir}
\end{align}
as claimed.

\paragraph{\textbf{(iv) Reverse direction and union bound}}
By the same argument,
$\sup_{g\in\mathscr G}\{R_{\mathrm{PN\mbox{-}TR}}(g)-\widehat R_{\mathrm{PN\mbox{-}TR}}(g)\}$ is also bounded by the same right-hand side with probability at least $1-\delta/2$.
Therefore, by the union bound, \eqref{eq:pntr-unif-bound-jp} holds with probability at least $1-\delta$.
\end{proof}

\subsection{Proof of Theorem~\ref{thm:pntr-estimation}}
\label{app:proof-thm-pntr-estimation}
\begin{proof}
Since $\widehat g_{\mathrm{PN\mbox{-}TR}}$ is an empirical risk minimizer, we have
$\widehat R_{\mathrm{PN\mbox{-}TR}}(\widehat g_{\mathrm{PN\mbox{-}TR}})
\le \widehat R_{\mathrm{PN\mbox{-}TR}}(g^*)$.
Moreover, by Lemma~\ref{lem:pntr-unif-jp}, the following holds with probability at least $1-\delta$:
\begin{align*}
& R_{\mathrm{PN\mbox{-}TR}}(\widehat g_{\mathrm{PN\mbox{-}TR}})-R_{\mathrm{PN\mbox{-}TR}}(g^*)
\\
&=
\Big(R_{\mathrm{PN\mbox{-}TR}}(\widehat g_{\mathrm{PN\mbox{-}TR}})-\widehat R_{\mathrm{PN\mbox{-}TR}}(\widehat g_{\mathrm{PN\mbox{-}TR}})\Big)
+
\Big(\widehat R_{\mathrm{PN\mbox{-}TR}}(\widehat g_{\mathrm{PN\mbox{-}TR}})-\widehat R_{\mathrm{PN\mbox{-}TR}}(g^*)\Big)
\\ &+
\Big(\widehat R_{\mathrm{PN\mbox{-}TR}}(g^*)-R_{\mathrm{PN\mbox{-}TR}}(g^*)\Big)
\\
&\le
\sup_{g\in\mathscr G}\big(R_{\mathrm{PN\mbox{-}TR}}(g)-\widehat R_{\mathrm{PN\mbox{-}TR}}(g)\big)
+0+
\sup_{g\in\mathscr G}\big(\widehat R_{\mathrm{PN\mbox{-}TR}}(g)-R_{\mathrm{PN\mbox{-}TR}}(g)\big)
\\
&\le
2\sup_{g\in\mathscr G}\left|\widehat R_{\mathrm{PN\mbox{-}TR}}(g)-R_{\mathrm{PN\mbox{-}TR}}(g)\right| \\
 & \le\;
4\bigl(L_\ell+4\beta L_{\mathrm{KL}}\bigr)\Big(
\pi_{\mathrm P}\mathfrak{R}_{n_{\mathrm P},p_{\mathrm P}}(\mathscr G)
+
\pi_{\mathrm N}\mathfrak{R}_{n_{\mathrm N},p_{\mathrm N}}(\mathscr G)
\Big) \\
 & +
8\beta L_{\mathrm{KL}}\,\varepsilon W d^{1/q}\left(
\frac{\pi_{\mathrm P}}{\sqrt{n_{\mathrm P}}}
+
\frac{\pi_{\mathrm N}}{\sqrt{n_{\mathrm N}}}
\right) \\
&+
\sqrt{2\ln\frac{2}{\delta}}\,(C_\ell+\beta C_{\mathrm{KL}})
\left(
\frac{\pi_{\mathrm P}}{\sqrt{n_{\mathrm P}}}
+
\frac{\pi_{\mathrm N}}{\sqrt{n_{\mathrm N}}}
\right).
\end{align*}
Thus, the claim of Theorem~\ref{thm:pntr-estimation} follows.
\end{proof}

\subsection{Proof of the Lemma ``Uniform Deviation Bound for uPU+TRADES'' (Lemma~\ref{lem:uputr-unif-jp})}
\label{app:proof-lem-uputr-unif-jp}
\begin{proof}
As in the previous section,
(i) we first show concentration of the uniform deviation using McDiarmid's inequality,
(ii) evaluate the expectation via ghost sampling and Rademacher complexity,
(iii) combine the results of (i) and (ii) to obtain a one-sided upper bound,
and (iv) combine the reverse direction via a union bound to obtain the uniform deviation bound (the lemma).
Finally, in (v), we derive the estimation error bound (the theorem) by the standard decomposition for an empirical risk minimizer.

\paragraph{\textbf{(i) McDiarmid's inequality}}
Since $\ell(\cdot)\le C_\ell$ and $\ell_{\mathrm{KL}}(\cdot)\le C_{\mathrm{KL}}$,
replacing one sample on the P side changes $\widehat R_{\mathrm{uPU\mbox{-}TR}}(g)$ by at most
$\frac{\pi_{\mathrm P}}{n_{\mathrm P}}(2C_\ell+\beta C_{\mathrm{KL}})$.
Similarly, replacing one sample on the U side changes $\widehat R_{\mathrm{uPU\mbox{-}TR}}(g)$ by at most
$
\frac{1}{n_{\mathrm U}}(C_\ell+\beta C_{\mathrm{KL}})
$.
Therefore, by McDiarmid's inequality, for any $t>0$,
\begin{align}
\Pr\!\Bigg(
&\sup_{g\in\mathscr G}\Big\{\widehat R_{\mathrm{uPU\mbox{-}TR}}(g)-R_{\mathrm{uPU\mbox{-}TR}}(g)\Big\}
-\mathbb E\Big[\sup_{g\in\mathscr G}\Big\{\widehat R_{\mathrm{uPU\mbox{-}TR}}(g)-R_{\mathrm{uPU\mbox{-}TR}}(g)\Big\}\Big]
\ge t
\Bigg)
\nonumber\\
&\le
\exp\!\Bigg(
-\frac{2t^2}{
\frac{\pi_{\mathrm P}^2(2C_\ell+\beta C_{\mathrm{KL}})^2}{n_{\mathrm P}}
+\frac{(C_\ell+\beta C_{\mathrm{KL}})^2}{n_{\mathrm U}}
}
\Bigg).
\label{eq:uputr-mcdiarmid-jp}
\end{align}

Setting the right-hand side equal to $\delta/2$, we obtain, with probability at least $1-\delta/2$,
\begin{align}
\sup_{g\in\mathscr G}\{\widehat R_{\mathrm{uPU\mbox{-}TR}}(g)-R_{\mathrm{uPU\mbox{-}TR}}(g)\}
\le\
&\mathbb E\Big[\sup_{g\in\mathscr G}\{\widehat R_{\mathrm{uPU\mbox{-}TR}}(g)-R_{\mathrm{uPU\mbox{-}TR}}(g)\}\Big]
\nonumber\\
&+
\sqrt{\frac{1}{2}\ln\frac{2}{\delta}}\,
\sqrt{
\frac{\pi_{\mathrm P}^2(2C_\ell+\beta C_{\mathrm{KL}})^2}{n_{\mathrm P}}
+\frac{(C_\ell+\beta C_{\mathrm{KL}})^2}{n_{\mathrm U}}
}\,.
\label{eq:uputr-mcdiarmid-root-jp}
\end{align}
Furthermore, by using the subadditivity of the square root, we obtain
\begin{align}
\sup_{g\in\mathscr G}\{\widehat R_{\mathrm{uPU\mbox{-}TR}}(g)-R_{\mathrm{uPU\mbox{-}TR}}(g)\}
\le\
&\mathbb E\Big[\sup_{g\in\mathscr G}\{\widehat R_{\mathrm{uPU\mbox{-}TR}}(g)-R_{\mathrm{uPU\mbox{-}TR}}(g)\}\Big]
\nonumber\\
&+
\sqrt{\frac{1}{2}\ln\frac{2}{\delta}}\,
\left(
\frac{\pi_{\mathrm P}(2C_\ell+\beta C_{\mathrm{KL}})}{\sqrt{n_{\mathrm P}}}
+\frac{C_\ell+\beta C_{\mathrm{KL}}}{\sqrt{n_{\mathrm U}}}
\right)
\label{eq:uputr-mcdiarmid-root2-jp}
\end{align}
as desired.

\paragraph{\textbf{(ii) Ghost sampling}}
Here, we upper bound
$\mathbb{E}\big[\sup_{g\in\mathscr G}\{\widehat R_{\mathrm{uPU\mbox{-}TR}}(g)-R_{\mathrm{uPU\mbox{-}TR}}(g)\}\big]$
by the Rademacher complexity.
The expectation in
$\mathbb{E}[\sup_{g\in\mathscr G}\widehat R_{\mathrm{uPU\mbox{-}TR}}(g)-R_{\mathrm{uPU\mbox{-}TR}}(g)]$
is taken over repeated sampling of
$(\mathscr X_{\mathrm P},\mathscr X_{\mathrm U})$ used to evaluate
$\widehat R_{\mathrm{uPU\mbox{-}TR}}(g)$.
That is,
$\mathbb{E}=\mathbb{E}_{(\mathscr X_{\mathrm P},\mathscr X_{\mathrm U})}$, and
$\widehat R_{\mathrm{uPU\mbox{-}TR}}(g)$ can also be written as
$\widehat R_{\mathrm{uPU\mbox{-}TR}}(g;\mathscr X_{\mathrm P},\mathscr X_{\mathrm U})$.
Introducing an independent ghost sample
$(\mathscr X_{\mathrm P}',\mathscr X_{\mathrm U}')$, we obtain
\begin{align}
&\mathbb{E}_{(\mathscr X_{\mathrm P},\mathscr X_{\mathrm U})}
\Big[\sup_{g\in\mathscr G}\widehat R_{\mathrm{uPU\mbox{-}TR}}(g)-R_{\mathrm{uPU\mbox{-}TR}}(g)\Big]
\nonumber\\
&=
\mathbb{E}_{(\mathscr X_{\mathrm P},\mathscr X_{\mathrm U})}
\Big[\sup_{g\in\mathscr G}\widehat R_{\mathrm{uPU\mbox{-}TR}}(g)
-\mathbb{E}_{(\mathscr X_{\mathrm P}',\mathscr X_{\mathrm U}')}
\widehat R_{\mathrm{uPU\mbox{-}TR}}(g;\mathscr X_{\mathrm P}',\mathscr X_{\mathrm U}')
\Big]
\nonumber\\
&\le
\mathbb{E}_{(\mathscr X_{\mathrm P},\mathscr X_{\mathrm U}),(\mathscr X_{\mathrm P}',\mathscr X_{\mathrm U}')}
\Big[
\sup_{g\in\mathscr G}
\widehat R_{\mathrm{uPU\mbox{-}TR}}(g;\mathscr X_{\mathrm P},\mathscr X_{\mathrm U})
-
\widehat R_{\mathrm{uPU\mbox{-}TR}}(g;\mathscr X_{\mathrm P}',\mathscr X_{\mathrm U}')
\Big]
\label{eq:uputr-ghost-sym}
\end{align}
The last inequality follows from Jensen's inequality, since $\sup$ is convex.

Next, we decompose the difference:
\[
\phi_{\mathrm P}(g,\bm x)
:=\ell\big(g(\bm x),+1\big)-\ell\big(g(\bm x),-1\big)+\beta\psi(g,\bm x),\qquad
\phi_{\mathrm U}(g,\bm x)
:=\ell\big(g(\bm x),-1\big)+\beta\psi(g,\bm x),
\]
where
\[
\psi(g,\bm x)
:=\max_{\|\bm\eta\|_\infty\le\varepsilon}
\ell_{\mathrm{KL}}\big(g(\bm x+\bm\eta),g(\bm x)\big).
\]
Then,
\begin{align}
&\widehat R_{\mathrm{uPU\mbox{-}TR}}(g;\mathscr X_{\mathrm P},\mathscr X_{\mathrm U})
-\widehat R_{\mathrm{uPU\mbox{-}TR}}(g;\mathscr X_{\mathrm P}',\mathscr X_{\mathrm U}')
\nonumber\\
&=
\frac{\pi_{\mathrm P}}{n_{\mathrm P}}
\sum_{i=1}^{n_{\mathrm P}}
\big(\phi_{\mathrm P}(g,\bm x_i^{\mathrm P})-\phi_{\mathrm P}(g,\bm x_i^{\mathrm P\prime})\big)
+
\frac{1}{n_{\mathrm U}}
\sum_{i=1}^{n_{\mathrm U}}
\big(\phi_{\mathrm U}(g,\bm x_i^{\mathrm U})-\phi_{\mathrm U}(g,\bm x_i^{\mathrm U\prime})\big).
\label{eq:uputr-diff-decomp}
\end{align}
Therefore, by the subadditivity of $\sup$,
\begin{align}
&\mathbb{E}_{(\mathscr X_{\mathrm P},\mathscr X_{\mathrm U}),(\mathscr X_{\mathrm P}',\mathscr X_{\mathrm U}')}
\Big[
\sup_{g\in\mathscr G}
\widehat R_{\mathrm{uPU\mbox{-}TR}}(g;\mathscr X_{\mathrm P},\mathscr X_{\mathrm U})
-
\widehat R_{\mathrm{uPU\mbox{-}TR}}(g;\mathscr X_{\mathrm P}',\mathscr X_{\mathrm U}')
\Big]
\nonumber\\
&\le
\frac{\pi_{\mathrm P}}{n_{\mathrm P}}
\mathbb{E}_{\mathscr X_{\mathrm P},\mathscr X_{\mathrm P}'}
\Big[\sup_{g\in\mathscr G}\sum_{i=1}^{n_{\mathrm P}}
\big(\phi_{\mathrm P}(g,\bm x_i^{\mathrm P})-\phi_{\mathrm P}(g,\bm x_i^{\mathrm P\prime})\big)\Big]
\nonumber\\
&\quad+
\frac{1}{n_{\mathrm U}}
\mathbb{E}_{\mathscr X_{\mathrm U},\mathscr X_{\mathrm U}'}
\Big[\sup_{g\in\mathscr G}\sum_{i=1}^{n_{\mathrm U}}
\big(\phi_{\mathrm U}(g,\bm x_i^{\mathrm U})-\phi_{\mathrm U}(g,\bm x_i^{\mathrm U\prime})\big)\Big].
\label{eq:uputr-sup-subadd}
\end{align}

Next, we symmetrize the difference terms appearing on the right-hand side of \eqref{eq:uputr-sup-subadd} and bound them by the Rademacher complexity.
For example, consider the P side.
Since $\bm x_i^{\mathrm P}$ and $\bm x_i^{\mathrm P\prime}$ are independent and identically distributed (both from $p_{\mathrm P}$),
the two differences
$\phi_{\mathrm P}(g,\bm x_i^{\mathrm P})-\phi_{\mathrm P}(g,\bm x_i^{\mathrm P\prime})$
and
$\phi_{\mathrm P}(g,\bm x_i^{\mathrm P\prime})-\phi_{\mathrm P}(g,\bm x_i^{\mathrm P})$
have the same distribution.
Therefore, letting
$L_{2,n_{\mathrm P}}:=\sum_{i=2}^{n_{\mathrm P}}\big(\phi_{\mathrm P}(g,\bm x_i^{\mathrm P})-\phi_{\mathrm P}(g,\bm x_i^{\mathrm P\prime})\big)$, we have
\begin{align}
&\mathbb{E}_{\mathscr X_{\mathrm P},\mathscr X_{\mathrm P}'}
\Big[\sup_{g\in\mathscr G}\sum_{i=1}^{n_{\mathrm P}}
\big(\phi_{\mathrm P}(g,\bm x_i^{\mathrm P})-\phi_{\mathrm P}(g,\bm x_i^{\mathrm P\prime})\big)\Big]
\nonumber\\
&=
\frac12\mathbb{E}_{\mathscr X_{\mathrm P},\mathscr X_{\mathrm P}'}
\Big[\sup_{g\in\mathscr G}
\big(\phi_{\mathrm P}(g,\bm x_1^{\mathrm P})-\phi_{\mathrm P}(g,\bm x_1^{\mathrm P\prime})\big)+L_{2,n_{\mathrm P}}\Big]
\nonumber\\
&\quad+
\frac12\mathbb{E}_{\mathscr X_{\mathrm P},\mathscr X_{\mathrm P}'}
\Big[\sup_{g\in\mathscr G}
\big(\phi_{\mathrm P}(g,\bm x_1^{\mathrm P\prime})-\phi_{\mathrm P}(g,\bm x_1^{\mathrm P})\big)+L_{2,n_{\mathrm P}}\Big]
\nonumber\\
&=
\mathbb{E}_{\sigma_1,\mathscr X_{\mathrm P},\mathscr X_{\mathrm P}'}
\Big[\sup_{g\in\mathscr G}
\sigma_1\big(\phi_{\mathrm P}(g,\bm x_1^{\mathrm P})-\phi_{\mathrm P}(g,\bm x_1^{\mathrm P\prime})\big)+L_{2,n_{\mathrm P}}\Big].
\end{align}
Repeating the same argument $n_{\mathrm P}$ times, and using independent Rademacher variables $\bm\sigma=(\sigma_1,\ldots,\sigma_{n_{\mathrm P}})$, we obtain
\begin{align}
&\mathbb{E}_{\mathscr X_{\mathrm P},\mathscr X_{\mathrm P}'}
\Big[\sup_{g\in\mathscr G}\sum_{i=1}^{n_{\mathrm P}}
\big(\phi_{\mathrm P}(g,\bm x_i^{\mathrm P})-\phi_{\mathrm P}(g,\bm x_i^{\mathrm P\prime})\big)\Big]
\nonumber\\
&=
\mathbb{E}_{\bm\sigma,\mathscr X_{\mathrm P},\mathscr X_{\mathrm P}'}
\Big[\sup_{g\in\mathscr G}\sum_{i=1}^{n_{\mathrm P}}
\sigma_i\big(\phi_{\mathrm P}(g,\bm x_i^{\mathrm P})-\phi_{\mathrm P}(g,\bm x_i^{\mathrm P\prime})\big)\Big]
\nonumber\\
&\le
\mathbb{E}_{\bm\sigma,\mathscr X_{\mathrm P}}
\Big[\sup_{g\in\mathscr G}\sum_{i=1}^{n_{\mathrm P}}\sigma_i\,\phi_{\mathrm P}(g,\bm x_i^{\mathrm P})\Big]
+
\mathbb{E}_{\bm\sigma,\mathscr X_{\mathrm P}'}
\Big[\sup_{g\in\mathscr G}\sum_{i=1}^{n_{\mathrm P}}(-\sigma_i)\,\phi_{\mathrm P}(g,\bm x_i^{\mathrm P\prime})\Big]
\nonumber\\
&=
2\,\mathbb{E}_{\bm\sigma,\mathscr X_{\mathrm P}}
\Big[\sup_{g\in\mathscr G}\sum_{i=1}^{n_{\mathrm P}}\sigma_i\,\phi_{\mathrm P}(g,\bm x_i^{\mathrm P})\Big]
=
2n_{\mathrm P}\,\mathfrak{R}_{n_{\mathrm P},p_{\mathrm P}}(\phi_{\mathrm P}\circ\mathscr G).
\label{eq:uputr-rad-phiP}
\end{align}
Similarly, for the U side,
\begin{align}
\mathbb{E}_{\mathscr X_{\mathrm U},\mathscr X_{\mathrm U}'}
\Big[\sup_{g\in\mathscr G}\sum_{i=1}^{n_{\mathrm U}}
\big(\phi_{\mathrm U}(g,\bm x_i^{\mathrm U})-\phi_{\mathrm U}(g,\bm x_i^{\mathrm U\prime})\big)\Big]
\le
2n_{\mathrm U}\,\mathfrak{R}_{n_{\mathrm U},p_{\mathrm U}}(\phi_{\mathrm U}\circ\mathscr G)
\label{eq:uputr-rad-phiU}
\end{align}
is obtained.

Finally, we upper bound $\mathfrak{R}(\phi_{\mathrm P}\circ\mathscr G)$ and
$\mathfrak{R}(\phi_{\mathrm U}\circ\mathscr G)$ by
$\mathfrak{R}(\mathscr G)$.
By the subadditivity of $\sup$,
\[
\mathfrak{R}_{n,q}(\phi_{\mathrm P}\circ\mathscr G)
\le
\mathfrak{R}_{n,q}(\ell(\cdot,+1)\circ\mathscr G)
+
\mathfrak{R}_{n,q}(\ell(\cdot,-1)\circ\mathscr G),
\]
\[
\mathfrak{R}_{n,q}(\phi_{\mathrm U}\circ\mathscr G)
\le
\mathfrak{R}_{n,q}(\ell(\cdot,-1)\circ\mathscr G)
+\beta\,\mathfrak{R}_{n,q}(\psi\circ\mathscr G).
\]
For the classification loss term, the contraction lemma gives
\[
\mathfrak{R}_{n,q}(\ell(\cdot,y)\circ\mathscr G)
\le
L_\ell\,\mathfrak{R}_{n,q}(\mathscr G)
\qquad (y\in\mathcal Y).
\]
For the KL term, applying Lemma~\ref{lem:vecContract} to
$f_g(\bm x)=g(\bm x)$ and $g_{g,\bm\eta}(\bm x)=g(\bm x+\bm\eta)$,
and then using Lemma~\ref{lem:advRad}, we obtain
\begin{align}
\mathfrak{R}_{n,q}(\psi\circ\mathscr G)
&\le
2L_{\mathrm{KL}}
\Big(
\mathfrak{R}_{n,q}(\mathscr G)
+
\mathfrak{R}_{n,q}(\{\,\bm x\mapsto g(\bm x+\bm\eta):\ \|\bm\eta\|_\infty\le\varepsilon,\ g\in\mathscr G\,\})
\Big)
\nonumber\\
&\le
2L_{\mathrm{KL}}
\Big(
2\mathfrak{R}_{n,q}(\mathscr G)
+\frac{\varepsilon W d^{1/q}}{\sqrt n}
\Big).
\label{eq:uputr-rad-kl-bound}
\end{align}
Substituting these bounds into \eqref{eq:uputr-ghost-sym}--\eqref{eq:uputr-sup-subadd}, we obtain
\begin{align}
\mathbb E\!\left[\sup_{g\in\mathscr G}\{\widehat R_{\mathrm{uPU\mbox{-}TR}}(g)-R_{\mathrm{uPU\mbox{-}TR}}(g)\}\right]
\ \le\
&
2\pi_{\mathrm P}\,\mathfrak{R}_{n_{\mathrm P},p_{\mathrm P}}\big(\phi_{\mathrm P}\circ\mathscr G\big)
+
2\,\mathfrak{R}_{n_{\mathrm U},p_{\mathrm U}}\big(\phi_{\mathrm U}\circ\mathscr G\big)
\nonumber\\
\le\
&
4\pi_{\mathrm P}\big(L_\ell+2\beta L_{\mathrm{KL}}\big)\mathfrak{R}_{n_{\mathrm P},p_{\mathrm P}}(\mathscr G) \nonumber\\
&+
2\big(L_\ell+4\beta L_{\mathrm{KL}}\big)\mathfrak{R}_{n_{\mathrm U},p_{\mathrm U}}(\mathscr G)
\nonumber\\
&+
4\beta L_{\mathrm{KL}}\,\varepsilon W d^{1/q}
\left(
\frac{\pi_{\mathrm P}}{\sqrt{n_{\mathrm P}}}
+\frac{1}{\sqrt{n_{\mathrm U}}}
\right).
\label{eq:uputr-exp-sup-bound}
\end{align}
This completes the bound.
\paragraph{\textbf{(iii) Combining the results of (i) and (ii)}}
Substituting the expectation bound in (ii), i.e., \eqref{eq:uputr-exp-sup-bound}, into the McDiarmid inequality result in (i), i.e., \eqref{eq:uputr-mcdiarmid-root2-jp}, we obtain, with probability at least $1-\delta/2$,
\begin{align}
\sup_{g\in\mathscr G}\Big\{\widehat R_{\mathrm{uPU\mbox{-}TR}}(g)-R_{\mathrm{uPU\mbox{-}TR}}(g)\Big\}
\ \le\
&
4\pi_{\mathrm P}\big(L_\ell+2\beta L_{\mathrm{KL}}\big)\mathfrak{R}_{n_{\mathrm P},p_{\mathrm P}}(\mathscr G)
+
2\big(L_\ell+4\beta L_{\mathrm{KL}}\big)\mathfrak{R}_{n_{\mathrm U},p_{\mathrm U}}(\mathscr G)
\nonumber\\
&+
4\beta L_{\mathrm{KL}}\,\varepsilon W d^{1/q}
\left(
\frac{\pi_{\mathrm P}}{\sqrt{n_{\mathrm P}}}
+\frac{1}{\sqrt{n_{\mathrm U}}}
\right)
\nonumber\\
&+
\sqrt{\frac{1}{2}\ln\frac{2}{\delta}}
\left(
\frac{\pi_{\mathrm P}(2C_\ell+\beta C_{\mathrm{KL}})}{\sqrt{n_{\mathrm P}}}
+
\frac{C_\ell+\beta C_{\mathrm{KL}}}{\sqrt{n_{\mathrm U}}}
\right).
\label{eq:uputr-sup-one-dir}
\end{align}
This completes the one-sided bound.
\end{proof}

\subsection{Proof of Theorem~\ref{thm:uputr-estimation}}
\label{app:proof-thm-uputr-estimation}
\begin{proof}
Since $\widehat g_{\mathrm{uPU\mbox{-}TR}}$ is an empirical risk minimizer, we have
\[
\widehat R_{\mathrm{uPU\mbox{-}TR}}(\widehat g_{\mathrm{uPU\mbox{-}TR}})
\le
\widehat R_{\mathrm{uPU\mbox{-}TR}}(g^*)
\]
Moreover, on the event where Lemma~\ref{lem:uputr-unif-jp} holds (which occurs with probability at least $1-\delta$),
\begin{align*}
& R_{\mathrm{uPU\mbox{-}TR}}(\widehat g_{\mathrm{uPU\mbox{-}TR}})
-
R_{\mathrm{uPU\mbox{-}TR}}(g^*)
\\
&=
\Big(R_{\mathrm{uPU\mbox{-}TR}}(\widehat g_{\mathrm{uPU\mbox{-}TR}})-\widehat R_{\mathrm{uPU\mbox{-}TR}}(\widehat g_{\mathrm{uPU\mbox{-}TR}})\Big) \nonumber\\
&+
\Big(\widehat R_{\mathrm{uPU\mbox{-}TR}}(\widehat g_{\mathrm{uPU\mbox{-}TR}})-\widehat R_{\mathrm{uPU\mbox{-}TR}}(g^*)\Big) \nonumber\\
&+
\Big(\widehat R_{\mathrm{uPU\mbox{-}TR}}(g^*)-R_{\mathrm{uPU\mbox{-}TR}}(g^*)\Big)
 \nonumber\\
&\le
\sup_{g\in\mathscr G}\big(R_{\mathrm{uPU\mbox{-}TR}}(g)-\widehat R_{\mathrm{uPU\mbox{-}TR}}(g)\big)
+0+
\sup_{g\in\mathscr G}\big(\widehat R_{\mathrm{uPU\mbox{-}TR}}(g)-R_{\mathrm{uPU\mbox{-}TR}}(g)\big)
\\
&\le
2\sup_{g\in\mathscr G}\left|\widehat R_{\mathrm{uPU\mbox{-}TR}}(g)-R_{\mathrm{uPU\mbox{-}TR}}(g)\right|.
\end{align*}
Therefore, multiplying \eqref{eq:uputr-unif-bound-jp} by $2$ yields \eqref{eq:uputr-estimation-bound}.
\end{proof}

\subsection{Proof of the Lemma ``Uniform Deviation Bound for nnPU+TRADES'' (Lemma~\ref{lem:nnputr-unif-jp})}
\label{app:proof-lem-nnputr-unif-jp}
\begin{proof}
\paragraph{\textbf{(i) McDiarmid's inequality}}
Since $\ell(\cdot)\le C_\ell$ and $\ell_{\mathrm{KL}}(\cdot)\le C_{\mathrm{KL}}$,
replacing one sample on the P side changes the sum in the first line by at most
$\frac{\pi_{\mathrm P}}{n_{\mathrm P}}(C_\ell+\beta C_{\mathrm{KL}})$.
In addition, since the truncation term $\max\{0,\cdot\}$ in nnPU is $1$-Lipschitz,
it can be upper bounded by the change in its argument.
Inside the truncation term, $\ell(g(\bm x),-1)$ on the P side is replaced at only one point, so the change is at most
$\frac{\pi_{\mathrm P}}{n_{\mathrm P}}C_\ell$.
Therefore, replacing one P-side sample changes
$\widehat R_{\mathrm{nnPU\mbox{-}TR}}(g)$ by at most
$\frac{\pi_{\mathrm P}}{n_{\mathrm P}}(2C_\ell+\beta C_{\mathrm{KL}})$
in total.
Similarly, replacing one U-side sample affects $\ell(g(\bm x),-1)$ inside the truncation term and
$\psi$ in the third line at only one point, and hence changes
$\widehat R_{\mathrm{nnPU\mbox{-}TR}}(g)$ by at most
$\frac{1}{n_{\mathrm U}}(C_\ell+\beta C_{\mathrm{KL}})$.
Therefore, by McDiarmid's inequality, for any $t>0$,
\begin{align}
\Pr\!\Bigg(
&\sup_{g\in\mathscr G}\Big\{\widehat R_{\mathrm{nnPU\mbox{-}TR}}(g)-R_{\mathrm{nnPU\mbox{-}TR}}(g)\Big\}
-\mathbb E\Big[\sup_{g\in\mathscr G}\Big\{\widehat R_{\mathrm{nnPU\mbox{-}TR}}(g)-R_{\mathrm{nnPU\mbox{-}TR}}(g)\Big\}\Big]
\ge t
\Bigg)
\nonumber\\
&\le
\exp\!\Bigg(
-\frac{2t^2}{
\frac{\pi_{\mathrm P}^2(2C_\ell+\beta C_{\mathrm{KL}})^2}{n_{\mathrm P}}
+\frac{(C_\ell+\beta C_{\mathrm{KL}})^2}{n_{\mathrm U}}
}
\Bigg).
\label{eq:nnputr-mcdiarmid-jp}
\end{align}

Setting the right-hand side equal to $\delta/2$, we obtain, with probability at least $1-\delta/2$,
\begin{align}
\sup_{g\in\mathscr G}\{\widehat R_{\mathrm{nnPU\mbox{-}TR}}(g)-R_{\mathrm{nnPU\mbox{-}TR}}(g)\}
\le\
&\mathbb E\Big[\sup_{g\in\mathscr G}\{\widehat R_{\mathrm{nnPU\mbox{-}TR}}(g)-R_{\mathrm{nnPU\mbox{-}TR}}(g)\}\Big]
\nonumber\\
&+
\sqrt{\frac{1}{2}\ln\frac{2}{\delta}}\,
\sqrt{
\frac{\pi_{\mathrm P}^2(2C_\ell+\beta C_{\mathrm{KL}})^2}{n_{\mathrm P}}
+\frac{(C_\ell+\beta C_{\mathrm{KL}})^2}{n_{\mathrm U}}
}\,.
\label{eq:nnputr-mcdiarmid-root-jp}
\end{align}
Furthermore, by using the subadditivity of the square root, we obtain
\begin{align}
\sup_{g\in\mathscr G}\{\widehat R_{\mathrm{nnPU\mbox{-}TR}}(g)-R_{\mathrm{nnPU\mbox{-}TR}}(g)\}
\le\
&\mathbb E\Big[\sup_{g\in\mathscr G}\{\widehat R_{\mathrm{nnPU\mbox{-}TR}}(g)-R_{\mathrm{nnPU\mbox{-}TR}}(g)\}\Big]
\nonumber\\
&+
\sqrt{\frac{1}{2}\ln\frac{2}{\delta}}\,
\left(
\frac{\pi_{\mathrm P}(2C_\ell+\beta C_{\mathrm{KL}})}{\sqrt{n_{\mathrm P}}}
+\frac{C_\ell+\beta C_{\mathrm{KL}}}{\sqrt{n_{\mathrm U}}}
\right)
\label{eq:nnputr-mcdiarmid-root2-jp}
\end{align}
as desired.
\paragraph{\textbf{(ii) Ghost sampling}}
Here, we upper bound
$\mathbb{E}\big[\sup_{g\in\mathscr G}\{\widehat R_{\mathrm{nnPU\mbox{-}TR}}(g)-R_{\mathrm{nnPU\mbox{-}TR}}(g)\}\big]$
by the Rademacher complexity.
Unlike the uPU case,
$\widehat R_{\mathrm{nnPU\mbox{-}TR}}(g)$ contains the nnPU truncation term $\max\{0,\cdot\}$, and therefore,
in general, $R_{\mathrm{nnPU\mbox{-}TR}}(g)=\mathbb{E}[\widehat R_{\mathrm{nnPU\mbox{-}TR}}(g)]$ does not hold.
Thus, using the $1$-Lipschitz property of $\max\{0,\cdot\}$,
\[
|\max\{0,a\}-\max\{0,b\}|\le |a-b|,
\]
we first reduce the problem to a sum of differences between empirical and population averages, and then evaluate each term by ghost sampling and symmetrization, as in uPU+TRADES.

First, define
\[
\phi_{+}(g,\bm x):=\ell\big(g(\bm x),+1\big)+\beta\,\psi(g,\bm x),\qquad
\phi_{-}(g,\bm x):=\ell\big(g(\bm x),-1\big),\qquad
\phi_{\psi}(g,\bm x):=\psi(g,\bm x),
\]
where
\[
\psi(g,\bm x)
:=\max_{\|\bm\eta\|_\infty\le\varepsilon}
\ell_{\mathrm{KL}}\big(g(\bm x+\bm\eta),g(\bm x)\big).
\]
Then, writing the inside of the truncation term as
\begin{align*}
\widehat s(g)
:={}&
-\frac{\pi_{\mathrm P}}{n_{\mathrm P}}\sum_{i=1}^{n_{\mathrm P}}\phi_{-}\big(g,\bm x_i^{\mathrm P}\big)
+\frac{1}{n_{\mathrm U}}\sum_{i=1}^{n_{\mathrm U}}\phi_{-}\big(g,\bm x_i^{\mathrm U}\big),\\
s(g)
:={}&
-\pi_{\mathrm P}\,\mathbb E_{\mathrm P}\!\left[\phi_{-}(g,\bm x)\right]
+\mathbb E_{\mathrm U}\!\left[\phi_{-}(g,\bm x)\right],
\end{align*}
we have
\begin{align}
\widehat R_{\mathrm{nnPU\mbox{-}TR}}(g)-R_{\mathrm{nnPU\mbox{-}TR}}(g)
={}&
\pi_{\mathrm P}\left\{
\frac{1}{n_{\mathrm P}}\sum_{i=1}^{n_{\mathrm P}}\phi_{+}\big(g,\bm x_i^{\mathrm P}\big)
-\mathbb E_{\mathrm P}\!\left[\phi_{+}(g,\bm x)\right]
\right\}\nonumber\\
&+\beta\left\{
\frac{1}{n_{\mathrm U}}\sum_{i=1}^{n_{\mathrm U}}\phi_{\psi}\big(g,\bm x_i^{\mathrm U}\big)
-\mathbb E_{\mathrm U}\!\left[\phi_{\psi}(g,\bm x)\right]
\right\}
\nonumber\\
&\quad
+\max\{0,\widehat s(g)\}-\max\{0,s(g)\}
\nonumber\\
\le{}&
\pi_{\mathrm P}\left\{
\frac{1}{n_{\mathrm P}}\sum_{i=1}^{n_{\mathrm P}}\phi_{+}\big(g,\bm x_i^{\mathrm P}\big)
-\mathbb E_{\mathrm P}\!\left[\phi_{+}(g,\bm x)\right]
\right\}\nonumber\\
&+\beta\left\{
\frac{1}{n_{\mathrm U}}\sum_{i=1}^{n_{\mathrm U}}\phi_{\psi}\big(g,\bm x_i^{\mathrm U}\big)
-\mathbb E_{\mathrm U}\!\left[\phi_{\psi}(g,\bm x)\right]
\right\}\nonumber\\
&+|\widehat s(g)-s(g)|
\nonumber\\
\le{}&
\pi_{\mathrm P}\left\{
\frac{1}{n_{\mathrm P}}\sum_{i=1}^{n_{\mathrm P}}\phi_{+}\big(g,\bm x_i^{\mathrm P}\big)
-\mathbb E_{\mathrm P}\!\left[\phi_{+}(g,\bm x)\right]
\right\}\nonumber\\
&+\beta\left\{
\frac{1}{n_{\mathrm U}}\sum_{i=1}^{n_{\mathrm U}}\phi_{\psi}\big(g,\bm x_i^{\mathrm U}\big)
-\mathbb E_{\mathrm U}\!\left[\phi_{\psi}(g,\bm x)\right]
\right\}
\nonumber\\
&\quad
+\pi_{\mathrm P}\left|
\frac{1}{n_{\mathrm P}}\sum_{i=1}^{n_{\mathrm P}}\phi_{-}\big(g,\bm x_i^{\mathrm P}\big)
-\mathbb E_{\mathrm P}\!\left[\phi_{-}(g,\bm x)\right]
\right|\nonumber\\
&+\left|
\frac{1}{n_{\mathrm U}}\sum_{i=1}^{n_{\mathrm U}}\phi_{-}\big(g,\bm x_i^{\mathrm U}\big)
-\mathbb E_{\mathrm U}\!\left[\phi_{-}(g,\bm x)\right]
\right|.
\label{eq:nnputr-lip-max}
\end{align}
Therefore, by the subadditivity of $\sup$ and the triangle inequality,
\begin{align}
\mathbb E\Big[\sup_{g\in\mathscr G}\{\widehat R_{\mathrm{nnPU\mbox{-}TR}}(g)-R_{\mathrm{nnPU\mbox{-}TR}}(g)\}\Big]
\le{}
&\ \pi_{\mathrm P}\,
\mathbb E\Bigg[\sup_{g\in\mathscr G}
\left\{
\frac{1}{n_{\mathrm P}}\sum_{i=1}^{n_{\mathrm P}}\phi_{+}\big(g,\bm x_i^{\mathrm P}\big)
-\mathbb E_{\mathrm P}\!\left[\phi_{+}(g,\bm x)\right]
\right\}\Bigg]
\nonumber\\
&+\beta\,
\mathbb E\Bigg[\sup_{g\in\mathscr G}
\left\{
\frac{1}{n_{\mathrm U}}\sum_{i=1}^{n_{\mathrm U}}\phi_{\psi}\big(g,\bm x_i^{\mathrm U}\big)
-\mathbb E_{\mathrm U}\!\left[\phi_{\psi}(g,\bm x)\right]
\right\}\Bigg]
\nonumber\\
&+\pi_{\mathrm P}\,
\mathbb E\Bigg[\sup_{g\in\mathscr G}
\left|
\frac{1}{n_{\mathrm P}}\sum_{i=1}^{n_{\mathrm P}}\phi_{-}\big(g,\bm x_i^{\mathrm P}\big)
-\mathbb E_{\mathrm P}\!\left[\phi_{-}(g,\bm x)\right]
\right|\Bigg]
\nonumber\\
&+
\mathbb E\Bigg[\sup_{g\in\mathscr G}
\left|
\frac{1}{n_{\mathrm U}}\sum_{i=1}^{n_{\mathrm U}}\phi_{-}\big(g,\bm x_i^{\mathrm U}\big)
-\mathbb E_{\mathrm U}\!\left[\phi_{-}(g,\bm x)\right]
\right|\Bigg].
\label{eq:nnputr-exp-sup-decomp}
\end{align}

Next, we evaluate each term by ghost sampling and symmetrization (as in uPU+TRADES).
For illustration, consider the first term on the P side.
Let
$\mathscr X_{\mathrm P}'=\{\bm x_i^{\mathrm P\prime}\}_{i=1}^{n_{\mathrm P}}$
be an independent ghost sample on the P side. Then,
\begin{align}
&\mathbb{E}_{\mathscr X_{\mathrm P}}
\Bigg[\sup_{g\in\mathscr G}
\left\{
\frac{1}{n_{\mathrm P}}\sum_{i=1}^{n_{\mathrm P}}\phi_{+}\big(g,\bm x_i^{\mathrm P}\big)
-\mathbb E_{\mathrm P}\!\left[\phi_{+}(g,\bm x)\right]
\right\}\Bigg]
\nonumber\\
&=
\mathbb{E}_{\mathscr X_{\mathrm P}}
\Bigg[\sup_{g\in\mathscr G}
\frac{1}{n_{\mathrm P}}\sum_{i=1}^{n_{\mathrm P}}\phi_{+}\big(g,\bm x_i^{\mathrm P}\big)
-\mathbb{E}_{\mathscr X_{\mathrm P}'}\Big\{\frac{1}{n_{\mathrm P}}\sum_{i=1}^{n_{\mathrm P}}\phi_{+}\big(g,\bm x_i^{\mathrm P\prime}\big)\Big\}
\Bigg]
\nonumber\\
&\le
\mathbb{E}_{\mathscr X_{\mathrm P},\mathscr X_{\mathrm P}'}
\Bigg[\sup_{g\in\mathscr G}
\frac{1}{n_{\mathrm P}}\sum_{i=1}^{n_{\mathrm P}}
\Big(\phi_{+}\big(g,\bm x_i^{\mathrm P}\big)-\phi_{+}\big(g,\bm x_i^{\mathrm P\prime}\big)\Big)\Bigg]
\label{eq:nnputr-ghost-P}
\end{align}
where the last inequality follows from Jensen's inequality, since $\sup$ is convex.

Here, since $\bm x_i^{\mathrm P}$ and $\bm x_i^{\mathrm P\prime}$ are independent and identically distributed (both from $p_{\mathrm P}$),
the two differences
$\phi_{+}(g,\bm x_i^{\mathrm P})-\phi_{+}(g,\bm x_i^{\mathrm P\prime})$
and
$\phi_{+}(g,\bm x_i^{\mathrm P\prime})-\phi_{+}(g,\bm x_i^{\mathrm P})$
have the same distribution. Therefore, letting
$L_{2,n_{\mathrm P}}:=\sum_{i=2}^{n_{\mathrm P}}\big(\phi_{+}(g,\bm x_i^{\mathrm P})-\phi_{+}(g,\bm x_i^{\mathrm P\prime})\big)$, we have
\begin{align}
&\mathbb{E}_{\mathscr X_{\mathrm P},\mathscr X_{\mathrm P}'}
\Big[\sup_{g\in\mathscr G}\sum_{i=1}^{n_{\mathrm P}}
\big(\phi_{+}(g,\bm x_i^{\mathrm P})-\phi_{+}(g,\bm x_i^{\mathrm P\prime})\big)\Big]
\nonumber\\
&=
\mathbb{E}_{\sigma_1,\mathscr X_{\mathrm P},\mathscr X_{\mathrm P}'}
\Big[\sup_{g\in\mathscr G}
\sigma_1\big(\phi_{+}(g,\bm x_1^{\mathrm P})-\phi_{+}(g,\bm x_1^{\mathrm P\prime})\big)+L_{2,n_{\mathrm P}}\Big].
\label{eq:nnputr-sym-stepP}
\end{align}
Repeating the same argument $n_{\mathrm P}$ times, and using independent Rademacher variables $\bm\sigma=(\sigma_1,\ldots,\sigma_{n_{\mathrm P}})$, we obtain
\begin{align}
&\mathbb{E}_{\mathscr X_{\mathrm P},\mathscr X_{\mathrm P}'}
\Big[\sup_{g\in\mathscr G}\sum_{i=1}^{n_{\mathrm P}}
\big(\phi_{+}(g,\bm x_i^{\mathrm P})-\phi_{+}(g,\bm x_i^{\mathrm P\prime})\big)\Big]
\nonumber\\
&=
\mathbb{E}_{\bm\sigma,\mathscr X_{\mathrm P},\mathscr X_{\mathrm P}'}
\Big[\sup_{g\in\mathscr G}\sum_{i=1}^{n_{\mathrm P}}
\sigma_i\big(\phi_{+}(g,\bm x_i^{\mathrm P})-\phi_{+}(g,\bm x_i^{\mathrm P\prime})\big)\Big]
\nonumber\\
&\le
\mathbb{E}_{\bm\sigma,\mathscr X_{\mathrm P}}
\Big[\sup_{g\in\mathscr G}\sum_{i=1}^{n_{\mathrm P}}
\sigma_i\,\phi_{+}(g,\bm x_i^{\mathrm P})\Big]
+
\mathbb{E}_{\bm\sigma,\mathscr X_{\mathrm P}'}
\Big[\sup_{g\in\mathscr G}\sum_{i=1}^{n_{\mathrm P}}
(-\sigma_i)\,\phi_{+}(g,\bm x_i^{\mathrm P\prime})\Big]
\nonumber\\
&=
2\,\mathbb{E}_{\bm\sigma,\mathscr X_{\mathrm P}}
\Big[\sup_{g\in\mathscr G}\sum_{i=1}^{n_{\mathrm P}}
\sigma_i\,\phi_{+}(g,\bm x_i^{\mathrm P})\Big]
=
2n_{\mathrm P}\,\mathfrak{R}_{n_{\mathrm P},p_{\mathrm P}}(\phi_{+}\circ\mathscr G).
\label{eq:nnputr-rad-phiPlusP}
\end{align}
Hence, combining this with \eqref{eq:nnputr-ghost-P}, we obtain
\[
\mathbb{E}_{\mathscr X_{\mathrm P}}
\Bigg[\sup_{g\in\mathscr G}
\left\{
\frac{1}{n_{\mathrm P}}\sum_{i=1}^{n_{\mathrm P}}\phi_{+}\big(g,\bm x_i^{\mathrm P}\big)
-\mathbb E_{\mathrm P}\!\left[\phi_{+}(g,\bm x)\right]
\right\}\Bigg]
\le
2\,\mathfrak{R}_{n_{\mathrm P},p_{\mathrm P}}(\phi_{+}\circ\mathscr G).
\]
Similarly, the second term on the U side and the third and fourth terms with absolute values can also be symmetrized, yielding
\begin{align}
\mathbb{E}\Bigg[\sup_{g\in\mathscr G}
\left\{
\frac{1}{n_{\mathrm U}}\sum_{i=1}^{n_{\mathrm U}}\phi_{\psi}\big(g,\bm x_i^{\mathrm U}\big)
-\mathbb E_{\mathrm U}\!\left[\phi_{\psi}(g,\bm x)\right]
\right\}\Bigg]
&\le
2\,\mathfrak{R}_{n_{\mathrm U},p_{\mathrm U}}(\phi_{\psi}\circ\mathscr G),
\nonumber\\
\mathbb{E}\Bigg[\sup_{g\in\mathscr G}
\left|
\frac{1}{n_{\mathrm P}}\sum_{i=1}^{n_{\mathrm P}}\phi_{-}\big(g,\bm x_i^{\mathrm P}\big)
-\mathbb E_{\mathrm P}\!\left[\phi_{-}(g,\bm x)\right]
\right|\Bigg]
&\le
2\,\mathfrak{R}_{n_{\mathrm P},p_{\mathrm P}}(\phi_{-}\circ\mathscr G),
\nonumber\\
\mathbb{E}\Bigg[\sup_{g\in\mathscr G}
\left|
\frac{1}{n_{\mathrm U}}\sum_{i=1}^{n_{\mathrm U}}\phi_{-}\big(g,\bm x_i^{\mathrm U}\big)
-\mathbb E_{\mathrm U}\!\left[\phi_{-}(g,\bm x)\right]
\right|\Bigg]
&\le
2\,\mathfrak{R}_{n_{\mathrm U},p_{\mathrm U}}(\phi_{-}\circ\mathscr G)
\label{eq:nnputr-rad-others}
\end{align}
Substituting these bounds into \eqref{eq:nnputr-exp-sup-decomp}, we obtain
\begin{align}
\mathbb E\Big[\sup_{g\in\mathscr G}\{\widehat R_{\mathrm{nnPU\mbox{-}TR}}(g)-R_{\mathrm{nnPU\mbox{-}TR}}(g)\}\Big]
\le{}
&\ 2\pi_{\mathrm P}\,\mathfrak{R}_{n_{\mathrm P},p_{\mathrm P}}(\phi_{+}\circ\mathscr G)\nonumber\\
&+2\beta\,\mathfrak{R}_{n_{\mathrm U},p_{\mathrm U}}(\phi_{\psi}\circ\mathscr G) \nonumber\\
&+2\pi_{\mathrm P}\,\mathfrak{R}_{n_{\mathrm P},p_{\mathrm P}}(\phi_{-}\circ\mathscr G) \nonumber\\
&+2\,\mathfrak{R}_{n_{\mathrm U},p_{\mathrm U}}(\phi_{-}\circ\mathscr G).
\label{eq:nnputr-rad-split}
\end{align}

Finally, we upper bound each Rademacher complexity term by $\mathfrak{R}(\mathscr G)$.
By the subadditivity of $\sup$,
\[
\mathfrak{R}_{n,p}(\phi_{+}\circ\mathscr G)
\le
\mathfrak{R}_{n,p}(\ell(\cdot,+1)\circ\mathscr G)
+\beta\,\mathfrak{R}_{n,p}(\psi\circ\mathscr G),\qquad
\mathfrak{R}_{n,p}(\phi_{-}\circ\mathscr G)
=
\mathfrak{R}_{n,p}(\ell(\cdot,-1)\circ\mathscr G).
\]
For the classification loss, the contraction lemma implies
\[
\mathfrak{R}_{n,p}(\ell(\cdot,y)\circ\mathscr G)\le L_{\ell}\,\mathfrak{R}_{n,p}(\mathscr G)\qquad(y\in\mathcal Y).
\]
For $\psi$, applying Lemma~\ref{lem:vecContract} and Lemma~\ref{lem:advRad}, we obtain
\begin{align}
\mathfrak{R}_{n,p}(\psi\circ\mathscr G)
&\le
2L_{\mathrm{KL}}
\Big(
\mathfrak{R}_{n,p}(\mathscr G)
+
\mathfrak{R}_{n,p}(\{\,\bm x\mapsto g(\bm x+\bm\eta):\ \|\bm\eta\|_\infty\le\varepsilon,\ g\in\mathscr G\,\})
\Big)
\nonumber\\
&\le
2L_{\mathrm{KL}}
\Big(
2\mathfrak{R}_{n,p}(\mathscr G)
+\frac{\varepsilon W d^{1/q}}{\sqrt n}
\Big).
\label{eq:nnputr-rad-kl-bound}
\end{align}
Substituting these bounds into \eqref{eq:nnputr-rad-split} and simplifying yields
\begin{align}
\mathbb E\Big[\sup_{g\in\mathscr G}\{\widehat R_{\mathrm{nnPU\mbox{-}TR}}(g)-R_{\mathrm{nnPU\mbox{-}TR}}(g)\}\Big]
\ \le\
&4\pi_{\mathrm P}\big(L_\ell+2\beta L_{\mathrm{KL}}\big)\mathfrak{R}_{n_{\mathrm P},p_{\mathrm P}}(\mathscr G)\nonumber\\
&+2\big(L_\ell+4\beta L_{\mathrm{KL}}\big)\mathfrak{R}_{n_{\mathrm U},p_{\mathrm U}}(\mathscr G)
\nonumber\\
&+
4\beta L_{\mathrm{KL}}\,\varepsilon W d^{1/q}
\left(\frac{\pi_{\mathrm P}}{\sqrt{n_{\mathrm P}}}+\frac{1}{\sqrt{n_{\mathrm U}}}\right).
\label{eq:nnputr-exp-sup-bound}
\end{align}
as desired.
\paragraph{\textbf{(iii) Combining the results of (i) and (ii)}}
Substituting the expectation bound in (ii), i.e., \eqref{eq:nnputr-exp-sup-bound}, into the McDiarmid inequality result in (i), i.e., \eqref{eq:nnputr-mcdiarmid-root2-jp}, we obtain, with probability at least $1-\delta/2$,
\begin{align}
\sup_{g\in\mathscr G}\Big\{\widehat R_{\mathrm{nnPU\mbox{-}TR}}(g)-R_{\mathrm{nnPU\mbox{-}TR}}(g)\Big\}
\ \le\
&4\pi_{\mathrm P}\big(L_\ell+2\beta L_{\mathrm{KL}}\big)\mathfrak{R}_{n_{\mathrm P},p_{\mathrm P}}(\mathscr G)
+2\big(L_\ell+4\beta L_{\mathrm{KL}}\big)\mathfrak{R}_{n_{\mathrm U},p_{\mathrm U}}(\mathscr G)
\nonumber\\
&+4\beta L_{\mathrm{KL}}\,\varepsilon W d^{1/q}
\left(
\frac{\pi_{\mathrm P}}{\sqrt{n_{\mathrm P}}}
+\frac{1}{\sqrt{n_{\mathrm U}}}
\right)
\nonumber\\
&+\sqrt{\frac{1}{2}\ln\frac{2}{\delta}}
\left(
\frac{\pi_{\mathrm P}(2C_\ell+\beta C_{\mathrm{KL}})}{\sqrt{n_{\mathrm P}}}
+\frac{C_\ell+\beta C_{\mathrm{KL}}}{\sqrt{n_{\mathrm U}}}
\right).
\label{eq:nnputr-sup-one-dir}
\end{align}
This establishes the one-sided bound.\\

\par\noindent\textbf{(iv) Opposite direction and union bound}\quad
Similarly,
$\sup_{g\in\mathscr G}\{R_{\mathrm{nnPU\mbox{-}TR}}(g)-\widehat R_{\mathrm{nnPU\mbox{-}TR}}(g)\}$
is also bounded by the right-hand side of \eqref{eq:nnputr-sup-one-dir}
with probability at least $1-\delta/2$.
Therefore, by the union bound, \eqref{eq:nnputr-unif-bound-jp} holds with probability at least $1-\delta$.
\end{proof}

\subsection{Proof of Theorem~\ref{thm:nnputr-estimation}}
\label{app:proof-thm-nnputr-estimation}
\begin{proof}
Since $\widehat g_{\mathrm{nnPU\mbox{-}TR}}$ is an empirical risk minimizer, we have
\[
\widehat R_{\mathrm{nnPU\mbox{-}TR}}(\widehat g_{\mathrm{nnPU\mbox{-}TR}})
\le
\widehat R_{\mathrm{nnPU\mbox{-}TR}}(g^*)
\]
Moreover, on the event where Lemma~\ref{lem:nnputr-unif-jp} holds (which occurs with probability at least $1-\delta$),
\begin{align*}
& R_{\mathrm{nnPU\mbox{-}TR}}(\widehat g_{\mathrm{nnPU\mbox{-}TR}})
-
R_{\mathrm{nnPU\mbox{-}TR}}(g^*)
\\
&=
\Big(R_{\mathrm{nnPU\mbox{-}TR}}(\widehat g_{\mathrm{nnPU\mbox{-}TR}})-\widehat R_{\mathrm{nnPU\mbox{-}TR}}(\widehat g_{\mathrm{nnPU\mbox{-}TR}})\Big) \nonumber\\
&+
\Big(\widehat R_{\mathrm{nnPU\mbox{-}TR}}(\widehat g_{\mathrm{nnPU\mbox{-}TR}})-\widehat R_{\mathrm{nnPU\mbox{-}TR}}(g^*)\Big) \nonumber\\
&+
\Big(\widehat R_{\mathrm{nnPU\mbox{-}TR}}(g^*)-R_{\mathrm{nnPU\mbox{-}TR}}(g^*)\Big)
\\
&\le
\sup_{g\in\mathscr G}\big(R_{\mathrm{nnPU\mbox{-}TR}}(g)-\widehat R_{\mathrm{nnPU\mbox{-}TR}}(g)\big)
+0+
\sup_{g\in\mathscr G}\big(\widehat R_{\mathrm{nnPU\mbox{-}TR}}(g)-R_{\mathrm{nnPU\mbox{-}TR}}(g)\big)
\\
&\le
2\sup_{g\in\mathscr G}\left|\widehat R_{\mathrm{nnPU\mbox{-}TR}}(g)-R_{\mathrm{nnPU\mbox{-}TR}}(g)\right|.
\end{align*}
Therefore, multiplying \eqref{eq:nnputr-unif-bound-jp} by $2$ yields \eqref{eq:nnputr-estimation-bound}.
\end{proof}

\subsection{Proof of the Theorem ``Condition on the Number of Unlabeled Samples for PU+TRADES to Outperform Supervised TRADES'' (Theorem~\ref{thm:nu-threshold-trades})}
\label{app:proof-thm-nu-threshold-trades}
\begin{proof}
(i) Comparing \eqref{eq:pntr-bound-simplified} and \eqref{eq:uputr-bound-simplified},
and canceling $\Gamma_\delta(\pi_{\mathrm P}/\sqrt{n_{\mathrm P}})$ from both sides, we obtain
\begin{align*}
\Gamma_\delta\frac{\pi_{\mathrm N}}{\sqrt{n_{\mathrm N}}}
\ >\
\frac{\pi_{\mathrm P}}{\sqrt{n_{\mathrm P}}}\Big(4L_\ell\,W C_x d^{1/q}+\kappa_\delta C_\ell\Big)
+\frac{\Gamma_\delta}{\sqrt{n_{\mathrm U}}}.
\end{align*}
Rearranging by moving the first term on the right-hand side, and solving for
$\Gamma_\delta/\sqrt{n_{\mathrm U}}$,
we obtain \eqref{eq:nu-threshold-uputr} under condition \eqref{eq:compare-pn-uputr-feasible}.
Part (ii) follows immediately from the fact that the bound for nnPU+TRADES has the same form as that for uPU+TRADES.
\end{proof}

\end{appendices}




%%===========================================================================================%%
%% If you are submitting to one of the Nature Portfolio journals, using the eJP submission   %%
%% system, please include the references within the manuscript file itself. You may do this  %%
%% by copying the reference list from your .bbl file, paste it into the main manuscript .tex %%
%% file, and delete the associated \verb+\bibliography+ commands.                            %%
%%===========================================================================================%%

\bibliography{sn-bibliography}% common bib file
%% if required, the content of .bbl file can be included here once bbl is generated
%%\input sn-article.bbl

\end{document}
